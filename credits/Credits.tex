\PassOptionsToPackage{unicode=true}{hyperref} % options for packages loaded elsewhere
\PassOptionsToPackage{hyphens}{url}
%
\documentclass[]{article}
\usepackage{lmodern}
\usepackage{amssymb,amsmath}
\usepackage{ifxetex,ifluatex}
\usepackage{fixltx2e} % provides \textsubscript
\ifnum 0\ifxetex 1\fi\ifluatex 1\fi=0 % if pdftex
  \usepackage[T1]{fontenc}
  \usepackage[utf8]{inputenc}
  \usepackage{textcomp} % provides euro and other symbols
\else % if luatex or xelatex
  \usepackage{unicode-math}
  \defaultfontfeatures{Ligatures=TeX,Scale=MatchLowercase}
\fi
% use upquote if available, for straight quotes in verbatim environments
\IfFileExists{upquote.sty}{\usepackage{upquote}}{}
% use microtype if available
\IfFileExists{microtype.sty}{%
\usepackage[]{microtype}
\UseMicrotypeSet[protrusion]{basicmath} % disable protrusion for tt fonts
}{}
\IfFileExists{parskip.sty}{%
\usepackage{parskip}
}{% else
\setlength{\parindent}{0pt}
\setlength{\parskip}{6pt plus 2pt minus 1pt}
}
\usepackage{hyperref}
\hypersetup{
            pdftitle={ISRaD Credits},
            pdfborder={0 0 0},
            breaklinks=true}
\urlstyle{same}  % don't use monospace font for urls
\usepackage[margin=1in]{geometry}
\usepackage{graphicx,grffile}
\makeatletter
\def\maxwidth{\ifdim\Gin@nat@width>\linewidth\linewidth\else\Gin@nat@width\fi}
\def\maxheight{\ifdim\Gin@nat@height>\textheight\textheight\else\Gin@nat@height\fi}
\makeatother
% Scale images if necessary, so that they will not overflow the page
% margins by default, and it is still possible to overwrite the defaults
% using explicit options in \includegraphics[width, height, ...]{}
\setkeys{Gin}{width=\maxwidth,height=\maxheight,keepaspectratio}
\setlength{\emergencystretch}{3em}  % prevent overfull lines
\providecommand{\tightlist}{%
  \setlength{\itemsep}{0pt}\setlength{\parskip}{0pt}}
\setcounter{secnumdepth}{0}
% Redefines (sub)paragraphs to behave more like sections
\ifx\paragraph\undefined\else
\let\oldparagraph\paragraph
\renewcommand{\paragraph}[1]{\oldparagraph{#1}\mbox{}}
\fi
\ifx\subparagraph\undefined\else
\let\oldsubparagraph\subparagraph
\renewcommand{\subparagraph}[1]{\oldsubparagraph{#1}\mbox{}}
\fi

% set default figure placement to htbp
\makeatletter
\def\fps@figure{htbp}
\makeatother


\title{ISRaD Credits}
\author{}
\date{\vspace{-2.5em}\textit{\today}}

\begin{document}
\maketitle

\hypertarget{main-compilations}{%
\section{Main compilations}\label{main-compilations}}

ISRaD has been built based on two main compilations:

\begin{itemize}
\tightlist
\item
  He, Trumbore, S. E., Torn, M. S., Harden, J. W., Vaughn, L. J. S.,
  Allison, S. D., \& Randerson, J. T. (2016). Radiocarbon constraints
  imply reduced carbon uptake by soils during the 21st century. Science,
  353(6306), 1419--1424. \url{https://doi.org/10.1126/science.aad4273}
\item
  Mathieu, Hatté, C., Balesdent, J., \& Parent, É. (2015). Deep soil
  carbon dynamics are driven more by soil type than by climate: a
  worldwide meta-analysis of radiocarbon profiles. Global Change
  Biology, 21(11), 4278--4292. Portico.
  \url{https://doi.org/10.1111/gcb.13012}
\end{itemize}

\hypertarget{studies-within-israd}{%
\section{Studies within ISRaD}\label{studies-within-israd}}

Currently there are 417 entries in ISRaD, which are from the following
publications:

\hypertarget{refs}{}
\leavevmode\hypertarget{ref-Abbott_1996}{}%
Abbott, M. B., \& Stafford, T. W. (1996). Radiocarbon geochemistry of
modern and ancient arctic lake systems, baffin island, canada.
\emph{Quaternary Research}, \emph{45}(3), 300--311.
\url{https://doi.org/10.1006/qres.1996.0031}

\leavevmode\hypertarget{ref-Agnelli_2002}{}%
Agnelli, A., Trumbore, S. E., Corti, G., \& Ugolini, F. C. (2002). The
dynamics of organic matter in rock fragments in soil investigated by 14
c dating and measurements of 13 c. \emph{European Journal of Soil
Science}, \emph{53}(1), 147--159.
\url{https://doi.org/10.1046/j.1365-2389.2002.00432.x}

\leavevmode\hypertarget{ref-Aiken_2014}{}%
Aiken, G. R., Spencer, R. G. M., Striegl, R. G., Schuster, P. F., \&
Raymond, P. A. (2014). Influences of glacier melt and permafrost thaw on
the age of dissolved organic carbon in the yukon river basin.
\emph{Global Biogeochemical Cycles}, \emph{28}(5), 525--537.
\url{https://doi.org/10.1002/2013gb004764}

\leavevmode\hypertarget{ref-Allard_1987}{}%
Allard, M., \& Seguin, M. K. (1987). The holocene evolution of
permafrost near the tree line, on the eastern coast of hudson bay
(northern quebec). \emph{Canadian Journal of Earth Sciences},
\emph{24}(11), 2206--2222. \url{https://doi.org/10.1139/e87-209}

\leavevmode\hypertarget{ref-Amon_2004}{}%
Amon, R. M., \& Meon, B. (2004). The biogeochemistry of dissolved
organic matter and nutrients in two large arctic estuaries and potential
implications for our understanding of the arctic ocean system.
\emph{Marine Chemistry}, \emph{92}(1-4), 311--330.
\url{https://doi.org/10.1016/j.marchem.2004.06.034}

\leavevmode\hypertarget{ref-Anderson_1975}{}%
Anderson, J., \& Muller, J. (1975). Palynological study of a holocene
peat and a miocene coal deposit from NW borneo. \emph{Review of
Palaeobotany and Palynology}, \emph{19}(4), 291--351.
\url{https://doi.org/10.1016/0034-6667(75)90049-4}

\leavevmode\hypertarget{ref-Andersson_2011}{}%
Andersson, R. A., Kuhry, P., Meyers, P., Zebühr, Y., Crill, P., \&
Mörth, M. (2011). Impacts of paleohydrological changes on n-alkane
biomarker compositions of a holocene peat sequence in the eastern
european russian arctic. \emph{Organic Geochemistry}, \emph{42}(9),
1065--1075. \url{https://doi.org/10.1016/j.orggeochem.2011.06.020}

\leavevmode\hypertarget{ref-Andreev_1997}{}%
Andreev, A., Klimanov, V., \& Sulerzhitsky, L. (1997). Younger dryas
pollen records from central and southern yakutia. \emph{Quaternary
International}, \emph{41-42}, 111--117.
\url{https://doi.org/10.1016/s1040-6182(96)00042-0}

\leavevmode\hypertarget{ref-Andreev_2001}{}%
Andreev, A., Klimanov, V., \& Sulerzhitsky, L. (2001). Vegetation and
climate history of the yana river lowland, russia, during the last
6400yr. \emph{Quaternary Science Reviews}, \emph{20}(1-3), 259--266.
\url{https://doi.org/10.1016/s0277-3791(00)00118-9}

\leavevmode\hypertarget{ref-Andreev_2004}{}%
Andreev, A., Tarasov, P., Klimanov, V., Melles, M., Lisitsyna, O., \&
Hubberten, H.-W. (2004). Vegetation and climate changes around the lama
lake, taymyr peninsula, russia during the late pleistocene and holocene.
\emph{Quaternary International}, \emph{122}(1), 69--84.
\url{https://doi.org/10.1016/j.quaint.2004.01.032}

\leavevmode\hypertarget{ref-Anshari_2004}{}%
Anshari, G., Kershaw, A. P., Kaars, S. V. D., \& Jacobsen, G. (2004).
Environmental change and peatland forest dynamics in the lake sentarum
area, west kalimantan, indonesia. \emph{Journal of Quaternary Science},
\emph{19}(7), 637--655. \url{https://doi.org/10.1002/jqs.879}

\leavevmode\hypertarget{ref-Anshari_2010}{}%
Anshari, G. Z., Afifudin, M., Nuriman, M., Gusmayanti, E., Arianie, L.,
Susana, R., Nusantara, R. W., Sugardjito, J., \& Rafiastanto, A. (2010).
Drainage and land use impacts on changes in selected peat properties and
peat degradation in west kalimantan province, indonesia.
\emph{Biogeosciences}, \emph{7}(11), 3403--3419.
\url{https://doi.org/10.5194/bg-7-3403-2010}

\leavevmode\hypertarget{ref-Walter_Anthony_2016}{}%
Anthony, K. W., Daanen, R., Anthony, P., Deimling, T. S. von, Ping,
C.-L., Chanton, J. P., \& Grosse, G. (2016). Methane emissions
proportional to permafrost carbon thawed in arctic lakes since the
1950s. \emph{Nature Geoscience}, \emph{9}(9), 679--682.
\url{https://doi.org/10.1038/ngeo2795}

\leavevmode\hypertarget{ref-Aravena_1993}{}%
Aravena, R., Warner, B. G., Charman, D. J., Belyea, L. R., Mathur, S.
P., \& Dinel, H. (1993). Carbon isotopic composition of deep carbon
gases in an ombrogenous peatland, northwestern ontario, canada.
\emph{Radiocarbon}, \emph{35}(2), 271--276.
\url{https://doi.org/10.1017/s0033822200064948}

\leavevmode\hypertarget{ref-Arlen_Pouliot_2005}{}%
Arlen-Pouliot, Y., \& Bhiry, N. (2005). Palaeoecology of a palsa and a
filled thermokarst pond in a permafrost peatland, subarctic québec,
canada. \emph{The Holocene}, \emph{15}(3), 408--419.
\url{https://doi.org/10.1191/0959683605hl818rp}

\leavevmode\hypertarget{ref-Atarashi_Andoh_2012}{}%
Atarashi-Andoh, M., Koarashi, J., Ishizuka, S., \& Hirai, K. (2012).
Seasonal patterns and control factors of CO2 effluxes from surface
litter, soil organic carbon, and root-derived carbon estimated using
radiocarbon signatures. \emph{Agricultural and Forest Meteorology},
\emph{152}, 149--158.
\url{https://doi.org/10.1016/j.agrformet.2011.09.015}

\leavevmode\hypertarget{ref-Baied_1993}{}%
Baied, C. A., \& Wheeler, J. C. (1993). Evolution of high andean puna
ecosystems: Environment, climate, and culture change over the last
12,000 years in the central andes. \emph{Mountain Research and
Development}, \emph{13}(2), 145. \url{https://doi.org/10.2307/3673632}

\leavevmode\hypertarget{ref-Baisden_2002}{}%
Baisden, W. T., Amundson, R., Cook, A. C., \& Brenner, D. L. (2002).
Turnover and storage of c and n in five density fractions from
california annual grassland surface soils. \emph{Global Biogeochemical
Cycles}, \emph{16}(4), 64--61--64--16.
\url{https://doi.org/10.1029/2001gb001822}

\leavevmode\hypertarget{ref-Baisden_2007}{}%
Baisden, W. T., \& Parfitt, R. L. (2007). Bomb 14C enrichment indicates
decadal c pool in deep soil? \emph{Biogeochemistry}, \emph{85}(1),
59--68. \url{https://doi.org/10.1007/s10533-007-9101-7}

\leavevmode\hypertarget{ref-Baisden_2011}{}%
Baisden, W. T., Parfitt, R. L., Ross, C., Schipper, L. A., \& Canessa,
S. (2011). Evaluating 50~years of time-series soil radiocarbon data:
Towards routine calculation of robust c residence times.
\emph{Biogeochemistry}, \emph{112}(1-3), 129--137.
\url{https://doi.org/10.1007/s10533-011-9675-y}

\leavevmode\hypertarget{ref-Basile_Doelsch_2005}{}%
Basile-Doelsch, I., Amundson, R., Stone, W. E. E., Masiello, C. A.,
Bottero, J. Y., Colin, F., Masin, F., Borschneck, D., \& Meunier, J. D.
(2005). Mineralogical control of organic carbon dynamics in a volcanic
ash soil on la reunion. \emph{European Journal of Soil Science},
\emph{0}(0), 050912034650042.
\url{https://doi.org/10.1111/j.1365-2389.2005.00703.x}

\leavevmode\hypertarget{ref-BAUER_2011}{}%
BAUER, I. E., \& VITT, D. H. (2011). Peatland dynamics in a complex
landscape: Development of a fen-bog complex in the sporadic
discontinuous permafrost zone of northern alberta, canada.
\emph{Boreas}, \emph{40}(4), 714--726.
\url{https://doi.org/10.1111/j.1502-3885.2011.00210.x}

\leavevmode\hypertarget{ref-Bauters_2019}{}%
Bauters, M., Vercleyen, O., Vanlauwe, B., Six, J., Bonyoma, B., Badjoko,
H., Hubau, W., Hoyt, A., Boudin, M., Verbeeck, H., \& Boeckx, P. (2019).
Long-term recovery of the functional community assembly and carbon pools
in an african tropical forest succession. \emph{Biotropica},
\emph{51}(3), 319--329. \url{https://doi.org/10.1111/btp.12647}

\leavevmode\hypertarget{ref-Beaulieu_Audy_2009}{}%
Beaulieu-Audy, V., Garneau, M., Richard, P. J., \& Asnong, H. (2009).
Holocene palaeoecological reconstruction of three boreal peatlands in
the la grande rivière region, québec, canada. \emph{The Holocene},
\emph{19}(3), 459--476. \url{https://doi.org/10.1177/0959683608101395}

\leavevmode\hypertarget{ref-Becker_Heidmann_2002}{}%
Becker-Heidmann, P., Andresen, O., Kalmar, D., Scharpenseel, H.-W., \&
Yaalon, D. H. (2002). Carbon dynamics in vertisols as revealed by
high-resolution sampling. \emph{Radiocarbon}, \emph{44}(1), 63--73.
\url{https://doi.org/10.1017/s0033822200064687}

\leavevmode\hypertarget{ref-Becker_Heidmann_1986}{}%
Becker-Heidmann, P., \& Scharpenseel, H.-W. (1986). Thin layer
\(\updelta\)13C and d14c monitoring of ``Lessive'' soil profiles.
\emph{Radiocarbon}, \emph{28}(2A), 383--390.
\url{https://doi.org/10.1017/s0033822200007499}

\leavevmode\hypertarget{ref-Becker_Heidmann_1989}{}%
Becker-Heidmann, P., \& Scharpenseel, H.-W. (1989). Carbon isotope
dynamics in some tropical soils. \emph{Radiocarbon}, \emph{31}(03),
672--679. \url{https://doi.org/10.1017/s0033822200012273}

\leavevmode\hypertarget{ref-Behling_1995}{}%
Behling, H. (1995). Investigations into the late pleistocene and
holocene history of vegetation and climate in santa catarina (s brazil).
\emph{Vegetation History and Archaeobotany}, \emph{4}(3).
\url{https://doi.org/10.1007/bf00203932}

\leavevmode\hypertarget{ref-Behling_2006}{}%
Behling, H., \& Pillar, V. D. (2006). Late quaternary vegetation,
biodiversity and fire dynamics on the southern brazilian highland and
their implication for conservation and management of modern araucaria
forest and grassland ecosystems. \emph{Philosophical Transactions of the
Royal Society B: Biological Sciences}, \emph{362}(1478), 243--251.
\url{https://doi.org/10.1098/rstb.2006.1984}

\leavevmode\hypertarget{ref-Beilman_2019}{}%
Beilman, D. W., Massa, C., Nichols, J. E., Timm, O. E., Kallstrom, R.,
\& Dunbar-Co, S. (2019). Dynamic holocene vegetation and north pacific
hydroclimate recorded in a mountain peatland, moloka`i, hawai`i.
\emph{Frontiers in Earth Science}, \emph{7}.
\url{https://doi.org/10.3389/feart.2019.00188}

\leavevmode\hypertarget{ref-van_Bellen_2011}{}%
Bellen, S. van, Garneau, M., \& Booth, R. K. (2011). Holocene carbon
accumulation rates from three ombrotrophic peatlands in boreal quebec,
canada: Impact of climate-driven ecohydrological change. \emph{The
Holocene}, \emph{21}(8), 1217--1231.
\url{https://doi.org/10.1177/0959683611405243}

\leavevmode\hypertarget{ref-Bellisario_1999}{}%
Bellisario, L. M., Bubier, J. L., Moore, T. R., \& Chanton, J. P.
(1999). Controls on CH4emissions from a northern peatland. \emph{Global
Biogeochemical Cycles}, \emph{13}(1), 81--91.
\url{https://doi.org/10.1029/1998gb900021}

\leavevmode\hypertarget{ref-Benfield_2021}{}%
Benfield, A. J., Yu, Z., \& Benavides, J. C. (2021). Environmental
controls over holocene carbon accumulation in distichia
muscoides-dominated peatlands in the eastern andes of colombia.
\emph{Quaternary Science Reviews}, \emph{251}, 106687.
\url{https://doi.org/10.1016/j.quascirev.2020.106687}

\leavevmode\hypertarget{ref-Benner_2004}{}%
Benner, R., Benitez-Nelson, B., Kaiser, K., \& Amon, R. M. W. (2004).
Export of young terrigenous dissolved organic carbon from rivers to the
arctic ocean. \emph{Geophysical Research Letters}, \emph{31}(5),
n/a--n/a. \url{https://doi.org/10.1029/2003gl019251}

\leavevmode\hypertarget{ref-Berg_2004}{}%
Berg, B., \& Gerstberger, P. (2004). Element fluxes with litterfall in
mature stands of norway spruce and european beech in bavaria, south
germany. In \emph{Ecological studies} (pp. 271--278). Springer Berlin
Heidelberg. \url{https://doi.org/10.1007/978-3-662-06073-5_16}

\leavevmode\hypertarget{ref-Berhe_2012}{}%
Berhe, A. A., Harden, J. W., Torn, M. S., Kleber, M., Burton, S. D., \&
Harte, J. (2012). Persistence of soil organic matter in eroding versus
depositional landform positions. \emph{Journal of Geophysical Research:
Biogeosciences}, \emph{117}(G2), n/a--n/a.
\url{https://doi.org/10.1029/2011jg001790}

\leavevmode\hypertarget{ref-Bhiry_2007}{}%
Bhiry, N., Payette, S., \& C. Robert. (2007). Peatland development at
the arctic tree line (québec, canada) influenced by flooding and
permafrost. \emph{Quaternary Research}, \emph{67}(3), 426--437.
\url{https://doi.org/10.1016/j.yqres.2006.11.009}

\leavevmode\hypertarget{ref-Biedenbender_2004}{}%
Biedenbender, S. H., McClaran, M. P., Quade, J., \& Weltz, M. A. (2004).
Landscape patterns of vegetation change indicated by soil carbon isotope
composition. \emph{Geoderma}, \emph{119}(1-2), 69--83.
\url{https://doi.org/10.1016/s0016-7061(03)00234-9}

\leavevmode\hypertarget{ref-Billings_1987}{}%
Billings, W. (1987). Carbon balance of alaskan tundra and taiga
ecosystems: Past, present and future. \emph{Quaternary Science Reviews},
\emph{6}(2), 165--177.
\url{https://doi.org/10.1016/0277-3791(87)90032-1}

\leavevmode\hypertarget{ref-Binkley_1999}{}%
Binkley, D., \& Resh, S. C. (1999). Rapid changes in soils following
eucalyptus afforestation in hawaii. \emph{Soil Science Society of
America Journal}, \emph{63}(1), 222--225.
\url{https://doi.org/10.2136/sssaj1999.03615995006300010032x}

\leavevmode\hypertarget{ref-Bird_2002}{}%
Bird, M., Santrùcková, H., Lloyd, J., \& Lawson, E. (2002). The isotopic
composition of soil organic carbon on a north-south transect in western
canada. \emph{European Journal of Soil Science}, \emph{53}(3), 393--403.
\url{https://doi.org/10.1046/j.1365-2389.2002.00444.x}

\leavevmode\hypertarget{ref-Blyakharchuk_2003}{}%
Blyakharchuk, T. A. (2003). Four new pollen sections tracing the
holocene vegetational development of the southern part of the west
siberan lowland. \emph{The Holocene}, \emph{13}(5), 715--731.
\url{https://doi.org/10.1191/0959683603hl658rp}

\leavevmode\hypertarget{ref-Blyakharchuk_1999}{}%
Blyakharchuk, T. A., \& Sulerzhitsky, L. D. (1999). Holocene
vegetational and climatic changes in the forest zone of western siberia
according to pollen records from the extrazonal palsa bog bugristoye.
\emph{The Holocene}, \emph{9}(5), 621--628.
\url{https://doi.org/10.1191/095968399676614561}

\leavevmode\hypertarget{ref-Bol_2003}{}%
Bol, R., Bolger, T., Cully, R., \& Little, D. (2003). Recalcitrant soil
organic materials mineralize more efficiently at higher temperatures.
\emph{Journal of Plant Nutrition and Soil Science}, \emph{166}(3),
300--307. \url{https://doi.org/10.1002/jpln.200390047}

\leavevmode\hypertarget{ref-BOL_1996}{}%
BOL, R., HUANG, Y., MERIDITH, J., EGLINTON, G., HARKNESS, D., \& INESON,
P. (1996). The 14C age and residence time of organic matter and its
lipid constituents in a stagnohumic gley soil. \emph{European Journal of
Soil Science}, \emph{47}(2), 215--222.
\url{https://doi.org/10.1111/j.1365-2389.1996.tb01392.x}

\leavevmode\hypertarget{ref-BORKEN_2005}{}%
BORKEN, W., SAVAGE, K., DAVIDSON, E. A., \& TRUMBORE, S. E. (2005).
Effects of experimental drought on soil respiration and radiocarbon
efflux from a temperate forest soil. \emph{Global Change Biology},
\emph{12}(2), 177--193.
\url{https://doi.org/10.1111/j.1365-2486.2005.001058.x}

\leavevmode\hypertarget{ref-Bouchard_2015}{}%
Bouchard, F., Laurion, I.,
Pr\&amp\(\mathsemicolon\)\#x0117\(\mathsemicolon\)skienis, V., Fortier,
D., Xu, X., \& Whiticar, M. J. (2015). Modern to millennium-old
greenhouse gases emitted from ponds and lakes of the eastern canadian
arctic (bylot island, nunavut). \emph{Biogeosciences}, \emph{12}(23),
7279--7298. \url{https://doi.org/10.5194/bg-12-7279-2015}

\leavevmode\hypertarget{ref-Bukombe_2021}{}%
Bukombe, B., Fiener, P., Hoyt, A. M., \& Doetterl, S. (2021).
\emph{Controls on heterotrophic soil respiration and carbon cycling in
geochemically distinct african tropical forest soils}.
\url{https://doi.org/10.5194/soil-2020-96}

\leavevmode\hypertarget{ref-Butman_2007}{}%
Butman, D., Raymond, P., Oh, N.-H., \& Mull, K. (2007). Quantity, 14C
age and lability of desorbed soil organic carbon in fresh water and
seawater. \emph{Organic Geochemistry}, \emph{38}(9), 1547--1557.
\url{https://doi.org/10.1016/j.orggeochem.2007.05.011}

\leavevmode\hypertarget{ref-Butnor_2017}{}%
Butnor, J. R., Samuelson, L. J., Johnsen, K. H., Anderson, P. H.,
Benecke, C. A. G., Boot, C. M., Cotrufo, M. F., Heckman, K. A., Jackson,
J. A., Stokes, T. A., \& Zarnoch, S. J. (2017). Vertical distribution
and persistence of soil organic carbon in fire-adapted longleaf pine
forests. \emph{Forest Ecology and Management}, \emph{390}, 15--26.
\url{https://doi.org/10.1016/j.foreco.2017.01.014}

\leavevmode\hypertarget{ref-De_Camargo_1999}{}%
Camargo, P. B. D., Trumbore, S. E., Martinelli, L. A., Davidson, E. A.,
Nepstad, D. C., \& Victoria, R. L. (1999). Soil carbon dynamics in
regrowing forest of eastern amazonia. \emph{Global Change Biology},
\emph{5}(6), 693--702.
\url{https://doi.org/10.1046/j.1365-2486.1999.00259.x}

\leavevmode\hypertarget{ref-Camill_2009}{}%
Camill, P., Barry, A., Williams, E., Andreassi, C., Limmer, J., \&
Solick, D. (2009). Climate-vegetation-fire interactions and their impact
on long-term carbon dynamics in a boreal peatland landscape in northern
manitoba, canada. \emph{Journal of Geophysical Research},
\emph{114}(G4). \url{https://doi.org/10.1029/2009jg001071}

\leavevmode\hypertarget{ref-Caner_2003}{}%
Caner, L., Toutain, F., Bourgeon, G., \& Herbillon, A.-J. (2003).
Occurrence of sombric-like subsurface a horizons in some andic soils of
the nilgiri hills (southern india) and their palaeoecological
significance. \emph{Geoderma}, \emph{117}(3-4), 251--265.
\url{https://doi.org/10.1016/s0016-7061(03)00127-7}

\leavevmode\hypertarget{ref-Carbone_2016}{}%
Carbone, M. S., Richardson, A. D., Chen, M., Davidson, E. A., Hughes,
H., Savage, K. E., \& Hollinger, D. Y. (2016). Constrained partitioning
of autotrophic and heterotrophic respiration reduces model uncertainties
of forest ecosystem carbon fluxes but not stocks. \emph{Journal of
Geophysical Research: Biogeosciences}, \emph{121}(9), 2476--2492.
\url{https://doi.org/10.1002/2016jg003386}

\leavevmode\hypertarget{ref-Carbone_2011}{}%
Carbone, M. S., Still, C. J., Ambrose, A. R., Dawson, T. E., Williams,
A. P., Boot, C. M., Schaeffer, S. M., \& Schimel, J. P. (2011). Seasonal
and episodic moisture controls on plant and microbial contributions to
soil respiration. \emph{Oecologia}, \emph{167}(1), 265--278.
\url{https://doi.org/10.1007/s00442-011-1975-3}

\leavevmode\hypertarget{ref-Carbone_2008}{}%
Carbone, M. S., Winston, G. C., \& Trumbore, S. E. (2008). Soil
respiration in perennial grass and shrub ecosystems: Linking
environmental controls with plant and microbial sources on seasonal and
diel timescales. \emph{Journal of Geophysical Research: Biogeosciences},
\emph{113}(G2), n/a--n/a. \url{https://doi.org/10.1029/2007jg000611}

\leavevmode\hypertarget{ref-Castanha_2012}{}%
Castanha, C., Trumbore, S., \& Amundso, R. (2012). Mineral and organic
matter characterization of density fractions of basalt- and
granite-derived soils in montane california. In \emph{An introduction to
the study of mineralogy}. InTech. \url{https://doi.org/10.5772/36735}

\leavevmode\hypertarget{ref-Chabbi_2009}{}%
Chabbi, A., Kögel-Knabner, I., \& Rumpel, C. (2009). Stabilised carbon
in subsoil horizons is located in spatially distinct parts of the soil
profile. \emph{Soil Biology and Biochemistry}, \emph{41}(2), 256--261.
\url{https://doi.org/10.1016/j.soilbio.2008.10.033}

\leavevmode\hypertarget{ref-Chasar_2000}{}%
Chasar, L. S., Chanton, J. P., Glaser, P. H., Siegel, D. I., \& Rivers,
J. S. (2000). Radiocarbon and stable carbon isotopic evidence for
transport and transformation of dissolved organic carbon, dissolved
inorganic carbon, and CH4in a northern minnesota peatland. \emph{Global
Biogeochemical Cycles}, \emph{14}(4), 1095--1108.
\url{https://doi.org/10.1029/1999gb001221}

\leavevmode\hypertarget{ref-Chen_2021}{}%
Chen, L., Fang, K., Wei, B., Qin, S., Feng, X., Hu, T., Ji, C., \& Yang,
Y. (2021). Soil carbon persistence governed by plant input and mineral
protection at regional and global scales. \emph{Ecology Letters},
\emph{24}(5), 1018--1028. \url{https://doi.org/10.1111/ele.13723}

\leavevmode\hypertarget{ref-Chen_2002}{}%
Chen, Q., Sun, Y., Shen, C., Peng, S., Yi, W., Li, Z., \& Jiang, M.
(2002). Organic matter turnover rates and CO2 flux from organic matter
decomposition of mountain soil profiles in the subtropical area, south
china. \emph{CATENA}, \emph{49}(3), 217--229.
\url{https://doi.org/10.1016/s0341-8162(02)00044-9}

\leavevmode\hypertarget{ref-Cherkinsky_1996}{}%
Cherkinsky, A. E. (1996). 14C dating and soil organic matter dynamics in
arctic and subarctic ecosystems. \emph{Radiocarbon}, \emph{38}(2),
241--245. \url{https://doi.org/10.1017/s0033822200017616}

\leavevmode\hypertarget{ref-Chiti_2015}{}%
Chiti, T., Certini, G., Forte, C., Papale, D., \& Valentini, R. (2015).
Radiocarbon-based assessment of heterotrophic soil respiration in two
mediterranean forests. \emph{Ecosystems}, \emph{19}(1), 62--72.
\url{https://doi.org/10.1007/s10021-015-9915-4}

\leavevmode\hypertarget{ref-Chiti_2010}{}%
Chiti, T., Certini, G., Grieco, E., \& Valentini, R. (2010). The role of
soil in storing carbon in tropical rainforests: The case of ankasa park,
ghana. \emph{Plant and Soil}, \emph{331}(1-2), 453--461.
\url{https://doi.org/10.1007/s11104-009-0265-x}

\leavevmode\hypertarget{ref-Chiti_2017}{}%
Chiti, T., Dı'az-Pinés, E., Butterbach-Bahl, K., Marzaioli, F., \&
Valentini, R. (2017). Soil organic carbon changes following degradation
and conversion to cypress and tea plantations in a tropical mountain
forest in kenya. \emph{Plant and Soil}, \emph{422}(1-2), 527--539.
\url{https://doi.org/10.1007/s11104-017-3489-1}

\leavevmode\hypertarget{ref-Chiti_2009}{}%
Chiti, T., Neubert, R., Janssens, I., Certini, G., Yuste, J. C., \&
Sirignano, C. (2009). Radiocarbon dating reveals different past
managements of adjacent forest soils in the campine region, belgium.
\emph{Geoderma}, \emph{149}(1-2), 137--142.
\url{https://doi.org/10.1016/j.geoderma.2008.11.030}

\leavevmode\hypertarget{ref-Chiti_2018}{}%
Chiti, T., Rey, A., Jeffery, K., Lauteri, M., Mihindou, V., Malhi, Y.,
Marzaioli, F., White, L. J. T., \& Valentini, R. (2018). Contribution
and stability of forest-derived soil organic carbon during woody
encroachment in a tropical savanna. A case study in gabon. \emph{Biology
and Fertility of Soils}, \emph{54}(8), 897--907.
\url{https://doi.org/10.1007/s00374-018-1313-6}

\leavevmode\hypertarget{ref-Chorover_2004}{}%
Chorover, J., Amistadi, M. K., \& Chadwick, O. A. (2004). Surface charge
evolution of mineral-organic complexes during pedogenesis in hawaiian
basalt. \emph{Geochimica et Cosmochimica Acta}, \emph{68}(23),
4859--4876. \url{https://doi.org/10.1016/j.gca.2004.06.005}

\leavevmode\hypertarget{ref-Cobb_2017}{}%
Cobb, A. R., Hoyt, A. M., Gandois, L., Eri, J., Dommain, R., Salim, K.
A., Kai, F. M., Su'ut, N. S. H., \& Harvey, C. F. (2017). How temporal
patterns in rainfall determine the geomorphology and carbon fluxes of
tropical peatlands. \emph{Proceedings of the National Academy of
Sciences}, 201701090. \url{https://doi.org/10.1073/pnas.1701090114}

\leavevmode\hypertarget{ref-Cole_2015}{}%
Cole, L. E. S., Bhagwat, S. A., \& Willis, K. J. (2015). Long-term
disturbance dynamics and resilience of tropical peat swamp forests.
\emph{Journal of Ecology}, \emph{103}(1), 16--30.
\url{https://doi.org/10.1111/1365-2745.12329}

\leavevmode\hypertarget{ref-Conen_2008}{}%
Conen, F., Zimmermann, M., Leifeld, J., Seth, B., \& Alewell, C. (2008).
Relative stability of soil carbon revealed by shifts in
\(\updelta\)\&lt\(\mathsemicolon\)sup\&gt\(\mathsemicolon\)15\&lt\(\mathsemicolon\)/sup\&gt\(\mathsemicolon\)N
and c:N ratio. \emph{Biogeosciences}, \emph{5}(1), 123--128.
\url{https://doi.org/10.5194/bg-5-123-2008}

\leavevmode\hypertarget{ref-Cook_2018}{}%
Cook, S., Whelan, M. J., Evans, C. D., Gauci, V., Peacock, M., Garnett,
M. H., Kho, L. K., Teh, Y. A., \& Page, S. E. (2018). Fluvial organic
carbon fluxes from oil palm plantations on tropical peatland.
\emph{Biogeosciences}, \emph{15}(24), 7435--7450.
\url{https://doi.org/10.5194/bg-15-7435-2018}

\leavevmode\hypertarget{ref-Cooper_2017}{}%
Cooper, M. D. A., Estop-Aragonés, C., Fisher, J. P., Thierry, A.,
Garnett, M. H., Charman, D. J., Murton, J. B., Phoenix, G. K., Treharne,
R., Kokelj, S. V., Wolfe, S. A., Lewkowicz, A. G., Williams, M., \&
Hartley, I. P. (2017). Limited contribution of permafrost carbon to
methane release from thawing peatlands. \emph{Nature Climate Change},
\emph{7}(7), 507--511. \url{https://doi.org/10.1038/nclimate3328}

\leavevmode\hypertarget{ref-Crews_1995}{}%
Crews, T. E., Kitayama, K., Fownes, J. H., Riley, R. H., Herbert, D. A.,
Mueller-Dombois, D., \& Vitousek, P. M. (1995). Changes in soil
phosphorus fractions and ecosystem dynamics across a long chronosequence
in hawaii. \emph{Ecology}, \emph{76}(5), 1407--1424.
\url{https://doi.org/10.2307/1938144}

\leavevmode\hypertarget{ref-Crow_2014}{}%
Crow, S. E., Reeves, M., Schubert, O. S., \& Sierra, C. A. (2014).
Optimization of method to quantify soil organic matter dynamics and
carbon sequestration potential in volcanic ash soils.
\emph{Biogeochemistry}, \emph{123}(1-2), 27--47.
\url{https://doi.org/10.1007/s10533-014-0051-6}

\leavevmode\hypertarget{ref-Cusack_2012}{}%
Cusack, D. F., Chadwick, O. A., Hockaday, W. C., \& Vitousek, P. M.
(2012). Mineralogical controls on soil black carbon preservation.
\emph{Global Biogeochemical Cycles}, \emph{26}(2), n/a--n/a.
\url{https://doi.org/10.1029/2011gb004109}

\leavevmode\hypertarget{ref-CUSACK_2010}{}%
CUSACK, D. F., TORN, M. S., McDOWELL, W. H., \& SILVER, W. L. (2010).
The response of heterotrophic activity and carbon cycling to nitrogen
additions and warming in two tropical soils. \emph{Global Change
Biology}. \url{https://doi.org/10.1111/j.1365-2486.2009.02131.x}

\leavevmode\hypertarget{ref-Czimczik_2005}{}%
Czimczik, C. I., Schmidt, M. W. I., \& Schulze, E.-D. (2005). Effects of
increasing fire frequency on black carbon and organic matter in podzols
of siberian scots pine forests. \emph{European Journal of Soil Science},
\emph{56}(3), 417--428.
\url{https://doi.org/10.1111/j.1365-2389.2004.00665.x}

\leavevmode\hypertarget{ref-Czimczik_2007}{}%
Czimczik, C. I., \& Trumbore, S. E. (2007). Short-term controls on the
age of microbial carbon sources in boreal forest soils. \emph{Journal of
Geophysical Research: Biogeosciences}, \emph{112}(G3), n/a--n/a.
\url{https://doi.org/10.1029/2006jg000389}

\leavevmode\hypertarget{ref-Czimczik_2010}{}%
Czimczik, C. I., \& Welker, J. M. (2010). Radiocarbon content of CO2
respired from high arctic tundra in northwest greenland. \emph{Arctic,
Antarctic, and Alpine Research}, \emph{42}(3), 342--350.
\url{https://doi.org/10.1657/1938-4246-42.3.342}

\leavevmode\hypertarget{ref-van_Dam_1997}{}%
Dam, D. van, Breemen, N. van, \& Veldkamp, E. (1997).
\emph{Biogeochemistry}, \emph{39}(3), 343--375.
\url{https://doi.org/10.1023/a:1005880031579}

\leavevmode\hypertarget{ref-Dargie_2017}{}%
Dargie, G. C., Lewis, S. L., Lawson, I. T., Mitchard, E. T. A., Page, S.
E., Bocko, Y. E., \& Ifo, S. A. (2017). Age, extent and carbon storage
of the central congo basin peatland complex. \emph{Nature},
\emph{542}(7639), 86--90. \url{https://doi.org/10.1038/nature21048}

\leavevmode\hypertarget{ref-Desjardins_1994}{}%
Desjardins, T., Andreux, F., Volkoff, B., \& Cerri, C. (1994). Organic
carbon and 13C contents in soils and soil size-fractions, and their
changes due to deforestation and pasture installation in eastern
amazonia. \emph{Geoderma}, \emph{61}(1-2), 103--118.
\url{https://doi.org/10.1016/0016-7061(94)90013-2}

\leavevmode\hypertarget{ref-Doetterl_2012}{}%
Doetterl, S., Six, J., Wesemael, B. V., \& Oost, K. V. (2012). Carbon
cycling in eroding landscapes: Geomorphic controls on soil organic c
pool composition and c stabilization. \emph{Global Change Biology},
\emph{18}(7), 2218--2232.
\url{https://doi.org/10.1111/j.1365-2486.2012.02680.x}

\leavevmode\hypertarget{ref-Doetterl_2015}{}%
Doetterl, S., Stevens, A., Six, J., Merckx, R., Oost, K. V., Pinto, M.
C., Casanova-Katny, A., Muñoz, C., Boudin, M., Venegas, E. Z., \&
Boeckx, P. (2015). Soil carbon storage controlled by interactions
between geochemistry and climate. \emph{Nature Geoscience},
\emph{8}(10), 780--783. \url{https://doi.org/10.1038/ngeo2516}

\leavevmode\hypertarget{ref-Dommain_2020}{}%
Dommain, R., Andama, M., McDonough, M. M., Prado, N. A., Goldhammer, T.,
Potts, R., Maldonado, J. E., Nkurunungi, J. B., \& Campana, M. G.
(2020). The challenges of reconstructing tropical biodiversity with
sedimentary ancient DNA: A 2200-year-long metagenomic record from bwindi
impenetrable forest, uganda. \emph{Frontiers in Ecology and Evolution},
\emph{8}. \url{https://doi.org/10.3389/fevo.2020.00218}

\leavevmode\hypertarget{ref-Dommain_2015}{}%
Dommain, R., Cobb, A. R., Joosten, H., Glaser, P. H., Chua, A. F. L.,
Gandois, L., Kai, F.-M., Noren, A., Salim, K. A., Su\textquotesingleut,
N. S. H., \& Harvey, C. F. (2015). Forest dynamics and tip-up pools
drive pulses of high carbon accumulation rates in a tropical peat dome
in borneo (southeast asia). \emph{Journal of Geophysical Research:
Biogeosciences}, \emph{120}(4), 617--640.
\url{https://doi.org/10.1002/2014jg002796}

\leavevmode\hypertarget{ref-D_rr_1980}{}%
Dörr, H., \& Münnich, K. O. (1980). Carbon-14 and carbon-13 in soil co2.
\emph{Radiocarbon}, \emph{22}(3), 909--918.
\url{https://doi.org/10.1017/s0033822200010316}

\leavevmode\hypertarget{ref-D_rr_1986}{}%
Dörr, H., \& Münnich, K. O. (1986). Annual variations of the 14C content
of soil CO2. \emph{Radiocarbon}, \emph{28}(2A), 338--345.
\url{https://doi.org/10.1017/s0033822200007438}

\leavevmode\hypertarget{ref-D_rr_1989}{}%
Dörr, H., \& Münnich, K. O. (1989). Downward movement of soil organic
matter and its influence on trace-element transport (210Pb, 137Cs) in
the soil. \emph{Radiocarbon}, \emph{31}(03), 655--663.
\url{https://doi.org/10.1017/s003382220001225x}

\leavevmode\hypertarget{ref-Drake_2019}{}%
Drake, T. W., Oost, K. V., Barthel, M., Bauters, M., Hoyt, A. M.,
Podgorski, D. C., Six, J., Boeckx, P., Trumbore, S. E., Ntaboba, L. C.,
\& Spencer, R. G. M. (2019). Mobilization of aged and biolabile soil
carbon by tropical deforestation. \emph{Nature Geoscience},
\emph{12}(7), 541--546. \url{https://doi.org/10.1038/s41561-019-0384-9}

\leavevmode\hypertarget{ref-Dredge_2005}{}%
Dredge, L. A., \& Mott, R. J. (2005). Holocene pollen records and
peatland development, northeastern manitoba. \emph{Géographie Physique
et Quaternaire}, \emph{57}(1), 7--19.
\url{https://doi.org/10.7202/010328ar}

\leavevmode\hypertarget{ref-DUTTA_2006}{}%
DUTTA, K., SCHUUR, E. A. G., NEFF, J. C., \& ZIMOV, S. A. (2006).
Potential carbon release from permafrost soils of northeastern siberia.
\emph{Global Change Biology}, \emph{12}(12), 2336--2351.
\url{https://doi.org/10.1111/j.1365-2486.2006.01259.x}

\leavevmode\hypertarget{ref-D_mig_2008}{}%
Dümig, A., Schad, P., Rumpel, C., Dignac, M.-F., \& Kögel-Knabner, I.
(2008). Araucaria forest expansion on grassland in the southern
brazilian highlands as revealed by 14C and \(\updelta\)13C studies.
\emph{Geoderma}, \emph{145}(1-2), 143--157.
\url{https://doi.org/10.1016/j.geoderma.2007.06.005}

\leavevmode\hypertarget{ref-Elder_2018}{}%
Elder, C. D., Xu, X., Walker, J., Schnell, J. L., Hinkel, K. M.,
Townsend-Small, A., Arp, C. D., Pohlman, J. W., Gaglioti, B. V., \&
Czimczik, C. I. (2018). Greenhouse gas emissions from diverse arctic
alaskan lakes are dominated by young carbon. \emph{Nature Climate
Change}, \emph{8}(2), 166--171.
\url{https://doi.org/10.1038/s41558-017-0066-9}

\leavevmode\hypertarget{ref-Ellis_2004}{}%
Ellis, C. J., \& Rochefort, L. (2004). CENTURY-SCALE DEVELOPMENT OF
POLYGON-PATTERNED TUNDRA WETLAND, BYLOT ISLAND (73 n, 80 w).
\emph{Ecology}, \emph{85}(4), 963--978.
\url{https://doi.org/10.1890/02-0614}

\leavevmode\hypertarget{ref-ELLIS_2006}{}%
ELLIS, C. J., \& ROCHEFORT, L. (2006). Long-term sensitivity of a high
arctic wetland to holocene climate change. \emph{Journal of Ecology},
\emph{94}(2), 441--454.
\url{https://doi.org/10.1111/j.1365-2745.2005.01085.x}

\leavevmode\hypertarget{ref-Elzein_1995}{}%
Elzein, A., \& Balesdent, J. (1995). Mechanistic simulation of vertical
distribution of carbon concentrations and residence times in soils.
\emph{Soil Science Society of America Journal}, \emph{59}(5),
1328--1335.
\url{https://doi.org/10.2136/sssaj1995.03615995005900050019x}

\leavevmode\hypertarget{ref-Estop_Aragon_s_2018}{}%
Estop-Aragonés, C., Cooper, M. D., Fisher, J. P., Thierry, A., Garnett,
M. H., Charman, D. J., Murton, J. B., Phoenix, G. K., Treharne, R.,
Sanderson, N. K., Burn, C. R., Kokelj, S. V., Wolfe, S. A., Lewkowicz,
A. G., Williams, M., \& Hartley, I. P. (2018). Limited release of
previously-frozen c and increased new peat formation after thaw in
permafrost peatlands. \emph{Soil Biology and Biochemistry}, \emph{118},
115--129. \url{https://doi.org/10.1016/j.soilbio.2017.12.010}

\leavevmode\hypertarget{ref-Eusterhues_2003}{}%
Eusterhues, K., Rumpel, C., Kleber, M., \& Kögel-Knabner, I. (2003).
Stabilisation of soil organic matter by interactions with minerals as
revealed by mineral dissolution and oxidative degradation. \emph{Organic
Geochemistry}, \emph{34}(12), 1591--1600.
\url{https://doi.org/10.1016/j.orggeochem.2003.08.007}

\leavevmode\hypertarget{ref-Ewing_2006}{}%
Ewing, S. A., Sanderman, J., Baisden, W. T., Wang, Y., \& Amundson, R.
(2006). Role of large-scale soil structure in organic carbon turnover:
Evidence from california grassland soils. \emph{Journal of Geophysical
Research}, \emph{111}(G3). \url{https://doi.org/10.1029/2006jg000174}

\leavevmode\hypertarget{ref-Favilli_2008}{}%
Favilli, F., Egli, M., Cherubini, P., Sartori, G., Haeberli, W., \&
Delbos, E. (2008). Comparison of different methods of obtaining a
resilient organic matter fraction in alpine soils. \emph{Geoderma},
\emph{145}(3-4), 355--369.
\url{https://doi.org/10.1016/j.geoderma.2008.04.002}

\leavevmode\hypertarget{ref-Fekete_2020}{}%
Fekete, I., Berki, I., Lajtha, K., Trumbore, S., Francioso, O.,
Gioacchini, P., Montecchio, D., Várbı'ró, G., Béni, Makádi, M., Demeter,
I., Madarász, B., Juhos, K., \& Kotroczó, Z. (2020). How will a drier
climate change carbon sequestration in soils of the deciduous forests of
central europe? \emph{Biogeochemistry}, \emph{152}(1), 13--32.
\url{https://doi.org/10.1007/s10533-020-00728-w}

\leavevmode\hypertarget{ref-Fernandez_1993}{}%
Fernandez, I. J., Rustad, L. E., \& Lawrence, G. B. (1993). Estimating
total soil mass, nutrient content, and trace metals in soils under a low
elevation spruce-fir forest. \emph{Canadian Journal of Soil Science},
\emph{73}(3), 317--328. \url{https://doi.org/10.4141/cjss93-034}

\leavevmode\hypertarget{ref-De_Feudis_2019}{}%
Feudis, M. D., Cardelli, V., Massaccesi, L., Trumbore, S., Antisari, L.
V., Cocco, S., Corti, G., \& Agnelli, A. (2019). Small altitudinal
change and rhizosphere affect the SOM light fractions but not the heavy
fraction in european beech forest soil. \emph{CATENA}, \emph{181},
104091. \url{https://doi.org/10.1016/j.catena.2019.104091}

\leavevmode\hypertarget{ref-Fierer_2005}{}%
Fierer, N., Chadwick, O. A., \& Trumbore, S. E. (2005). Production of
CO2 in soil profiles of a california annual grassland.
\emph{Ecosystems}, \emph{8}(4), 412--429.
\url{https://doi.org/10.1007/s10021-003-0151-y}

\leavevmode\hypertarget{ref-Fillion_2014}{}%
Fillion, M.-È., Bhiry, N., \& Touazi, M. (2014). Differential
development of two palsa fields in a peatland located near
whapmagoostui-kuujjuarapik, northern québec, canada. \emph{Arctic,
Antarctic, and Alpine Research}, \emph{46}(1), 40--54.
\url{https://doi.org/10.1657/1938-4246-46.1.40}

\leavevmode\hypertarget{ref-Finstad_2020}{}%
Finstad, K., Straaten, O., Veldkamp, E., \& McFarlane, K. (2020). Soil
carbon dynamics following land use changes and conversion to oil palm
plantations in tropical lowlands inferred from radiocarbon. \emph{Global
Biogeochemical Cycles}, \emph{34}(9).
\url{https://doi.org/10.1029/2019gb006461}

\leavevmode\hypertarget{ref-Fontaine_2007}{}%
Fontaine, S., Barot, S., Barré, P., Bdioui, N., Mary, B., \& Rumpel, C.
(2007). Stability of organic carbon in deep soil layers controlled by
fresh carbon supply. \emph{Nature}, \emph{450}(7167), 277--280.
\url{https://doi.org/10.1038/nature06275}

\leavevmode\hypertarget{ref-de_Freitas_2001}{}%
Freitas, H. A. de, Pessenda, L. C. R., Aravena, R., Gouveia, S. E. M.,
Souza Ribeiro, A. de, \& Boulet, R. (2001). Late quaternary vegetation
dynamics in the southern amazon basin inferred from carbon isotopes in
soil organic matter. \emph{Quaternary Research}, \emph{55}(1), 39--46.
\url{https://doi.org/10.1006/qres.2000.2192}

\leavevmode\hypertarget{ref-Gandois_2014}{}%
Gandois, L., Teisserenc, R., Cobb, A., Chieng, H., Lim, L., Kamariah,
A., Hoyt, A., \& Harvey, C. (2014). Origin, composition, and
transformation of dissolved organic matter in tropical peatlands.
\emph{Geochimica et Cosmochimica Acta}, \emph{137}, 35--47.
\url{https://doi.org/10.1016/j.gca.2014.03.012}

\leavevmode\hypertarget{ref-Garneau_2007}{}%
Garneau, M. (2007). Analyses macrofossiles d'un dépot de tourbe dans la
région de hot weather creek, péninsule de fosheim, ı\^{}le d'Ellesmere,
territoires du nord-ouest. \emph{Géographie Physique et Quaternaire},
\emph{46}(3), 285--294. \url{https://doi.org/10.7202/032915ar}

\leavevmode\hypertarget{ref-Garneau_2014}{}%
Garneau, M., Bellen, S. van, Magnan, G., Beaulieu-Audy, V., Lamarre, A.,
\& Asnong, H. (2014). Holocene carbon dynamics of boreal and subarctic
peatlands from québec, canada. \emph{The Holocene}, \emph{24}(9),
1043--1053. \url{https://doi.org/10.1177/0959683614538076}

\leavevmode\hypertarget{ref-Gaudinski_2000}{}%
Gaudinski, J. B., Trumbore, S. E., Davidson, E. A., \& Zheng, S. (2000).
\emph{Biogeochemistry}, \emph{51}(1), 33--69.
\url{https://doi.org/10.1023/a:1006301010014}

\leavevmode\hypertarget{ref-Gentsch_2018}{}%
Gentsch, N., Wild, B., Mikutta, R., Čapek, P., Diáková, K., Schrumpf,
M., Turner, S., Minnich, C., Schaarschmidt, F., Shibistova, O.,
Schnecker, J., Urich, T., Gittel, A., Šantrůčková, H., Bárta, J.,
Lashchinskiy, N., Fuß, R., Richter, A., \& Guggenberger, G. (2018).
Temperature response of permafrost soil carbon is attenuated by mineral
protection. \emph{Global Change Biology}, \emph{24}(8), 3401--3415.
\url{https://doi.org/10.1111/gcb.14316}

\leavevmode\hypertarget{ref-Giardina_2014}{}%
Giardina, C. P., Litton, C. M., Crow, S. E., \& Asner, G. P. (2014).
Warming-related increases in soil CO2 efflux are explained by increased
below-ground carbon flux. \emph{Nature Climate Change}, \emph{4}(9),
822--827. \url{https://doi.org/10.1038/nclimate2322}

\leavevmode\hypertarget{ref-Gillson_2004}{}%
Gillson, L. (2004). Testing non-equilibrium theories in savannas: 1400
years of vegetation change in tsavo national park, kenya.
\emph{Ecological Complexity}, \emph{1}(4), 281--298.
\url{https://doi.org/10.1016/j.ecocom.2004.06.001}

\leavevmode\hypertarget{ref-GOH_1976}{}%
GOH, K. M., RAFTER, T. A., STOUT, J. D., \& WALKER, T. W. (1976). THE
ACCUMULATION OF SOIL ORGANIC MATTER AND ITS CARBON ISOTOPE CONTENT IN a
CHRONOSEQUENCE OF SOILS DEVELOPED ON AEOLIAN SAND IN NEW ZEALAND.
\emph{Journal of Soil Science}, \emph{27}(1), 89--100.
\url{https://doi.org/10.1111/j.1365-2389.1976.tb01979.x}

\leavevmode\hypertarget{ref-GOH_1977}{}%
GOH, K. M., STOUT, J. D., \& RAFTER, T. A. (1977). RADIOCARBON
ENRICHMENT OF SOIL ORGANIC MATTER FRACTIONS IN NEW ZEALAND SOILS.
\emph{Soil Science}, \emph{123}(6), 385--391.
\url{https://doi.org/10.1097/00010694-197706000-00007}

\leavevmode\hypertarget{ref-Gonz_lez_Dom_nguez_2019}{}%
González-Domı'nguez, B., Niklaus, P. A., Studer, M. S., Hagedorn, F.,
Wacker, L., Haghipour, N., Zimmermann, S., Walthert, L., McIntyre, C.,
\& Abiven, S. (2019). Temperature and moisture are minor drivers of
regional-scale soil organic carbon dynamics. \emph{Scientific Reports},
\emph{9}(1). \url{https://doi.org/10.1038/s41598-019-42629-5}

\leavevmode\hypertarget{ref-GUILLET_2001}{}%
GUILLET, B., ACHOUNDONG, G., HAPPI, J. Y., BEYALA, V. K. K., BONVALLOT,
J., RIERA, B., MARIOTTI, A., \& SCHWARTZ, D. (2001). Agreement between
floristic and soil organic carbon isotope (13C/12C, 14C) indicators of
forest invasion of savannas during the last century in cameroon.
\emph{Journal of Tropical Ecology}, \emph{17}(6), 809--832.
\url{https://doi.org/10.1017/s0266467401001614}

\leavevmode\hypertarget{ref-Guillet_1988}{}%
Guillet, B., Faivre, P., Mariotti, A., \& Khobzi, J. (1988). The 14C
dates and 13C/12C ratios of soil organic matter as a means of studying
the past vegetation in intertropical regions: Examples from colombia
(south america). \emph{Palaeogeography, Palaeoclimatology,
Palaeoecology}, \emph{65}(1-2), 51--58.
\url{https://doi.org/10.1016/0031-0182(88)90111-3}

\leavevmode\hypertarget{ref-Guo_2003}{}%
Guo, L., Lehner, J. K., White, D. M., \& Garland, D. (2003).
Heterogeneity of natural organic matter from the chena river, alaska.
\emph{Water Research}, \emph{37}(5), 1015--1022.
\url{https://doi.org/10.1016/s0043-1354(02)00443-8}

\leavevmode\hypertarget{ref-Guo_2006}{}%
Guo, L., \& Macdonald, R. W. (2006). Source and transport of terrigenous
organic matter in the upper yukon river: Evidence from isotope
(\(\updelta\)13C, \(\upDelta\)14C, and \(\updelta\)15N) composition of
dissolved, colloidal, and particulate phases. \emph{Global
Biogeochemical Cycles}, \emph{20}(2), n/a--n/a.
\url{https://doi.org/10.1029/2005gb002593}

\leavevmode\hypertarget{ref-Guo_2007}{}%
Guo, L., Ping, C.-L., \& Macdonald, R. W. (2007). Mobilization pathways
of organic carbon from permafrost to arctic rivers in a changing
climate. \emph{Geophysical Research Letters}, \emph{34}(13), n/a--n/a.
\url{https://doi.org/10.1029/2007gl030689}

\leavevmode\hypertarget{ref-Hall_2015}{}%
Hall, S. J., McNicol, G., Natake, T., \& Silver, W. L. (2015).
\emph{Large fluxes and rapid turnover of mineral-associated carbon
across topographic gradients in a humid tropical forest: Insights from
paired
\&lt\(\mathsemicolon\)sup\&gt\(\mathsemicolon\)14\&lt\(\mathsemicolon\)/sup\&gt\(\mathsemicolon\)C
analysis}. \url{https://doi.org/10.5194/bgd-12-891-2015}

\leavevmode\hypertarget{ref-Harden_2002}{}%
Harden, J., Fries, T., \& Pavich, M. (2002). \emph{Biogeochemistry},
\emph{60}(3), 317--336. \url{https://doi.org/10.1023/a:1020308729553}

\leavevmode\hypertarget{ref-Hardie_2009}{}%
Hardie, S., Garnett, M., Fallick, A., Ostle, N., \& Rowland, A. (2009).
Bomb-14C analysis of ecosystem respiration reveals that peatland
vegetation facilitates release of old carbon. \emph{Geoderma},
\emph{153}(3-4), 393--401.
\url{https://doi.org/10.1016/j.geoderma.2009.09.002}

\leavevmode\hypertarget{ref-Hardie_2011}{}%
Hardie, S., Garnett, M., Fallick, A., Rowland, A., Ostle, N., \&
Flowers, T. (2011). Abiotic drivers and their interactive effect on the
flux and carbon isotope (14C and \(\updelta\)13C) composition of
peat-respired CO2. \emph{Soil Biology and Biochemistry}, \emph{43}(12),
2432--2440. \url{https://doi.org/10.1016/j.soilbio.2011.08.010}

\leavevmode\hypertarget{ref-Harkness_1986}{}%
Harkness, D. D., Harrison, A. F., \& Bacon, P. J. (1986). The temporal
distribution of ``bomb'' 14C in a forest soil. \emph{Radiocarbon},
\emph{28}(2A), 328--337. \url{https://doi.org/10.1017/s0033822200007426}

\leavevmode\hypertarget{ref-Harris_1994}{}%
Harris, S. A., \& Schmidt, I. H. (1994). Permafrost aggradation and peat
accumulation since 1200 years b.p. In peat plateaus at tuchitua, yukon
territory (canada). \emph{Journal of Paleolimnology}, \emph{12}(1),
3--17. \url{https://doi.org/10.1007/bf00677986}

\leavevmode\hypertarget{ref-Hatton_2012}{}%
Hatton, P.-J., Kleber, M., Zeller, B., Moni, C., Plante, A. F.,
Townsend, K., Gelhaye, L., Lajtha, K., \& Derrien, D. (2012). Transfer
of litter-derived n to soil mineralorganic associations: Evidence from
decadal 15N tracer experiments. \emph{Organic Geochemistry},
\emph{42}(12), 1489--1501.
\url{https://doi.org/10.1016/j.orggeochem.2011.05.002}

\leavevmode\hypertarget{ref-https:ux2fux2fdoi.orgux2f10.5281ux2fzenodo.1486081}{}%
Heckman, K. A. (2010). \emph{Pedogenesis \&amp; carbon dynamics across a
lithosequence under ponderosa pine}.
\url{https://doi.org/10.5281/ZENODO.1486081}

\leavevmode\hypertarget{ref-Heckman_2018}{}%
Heckman, K., Lawrence, C. R., \& Harden, J. W. (2018). A sequential
selective dissolution method to quantify storage and stability of
organic carbon associated with al and fe hydroxide phases.
\emph{Geoderma}, \emph{312}, 24--35.
\url{https://doi.org/10.1016/j.geoderma.2017.09.043}

\leavevmode\hypertarget{ref-Hilton_2015}{}%
Hilton, R. G., Galy, V., Gaillardet, J., Dellinger, M., Bryant, C.,
O\textquotesingleRegan, M., Gröcke, D. R., Coxall, H., Bouchez, J., \&
Calmels, D. (2015). Erosion of organic carbon in the arctic as a
geological carbon dioxide sink. \emph{Nature}, \emph{524}(7563), 84--87.
\url{https://doi.org/10.1038/nature14653}

\leavevmode\hypertarget{ref-https:ux2fux2fdoi.orgux2f10.5281ux2fzenodo.3370057}{}%
Holden, S. R., Czimczik, C. I., Xu, X., \& Treseder, K. K. (2019).
\emph{Soil radiocarbon data from a fire chronosequence near delta
junction, alaska}. Zenodo. \url{https://doi.org/10.5281/ZENODO.3370057}

\leavevmode\hypertarget{ref-Holmquist_2014}{}%
Holmquist, J. R., MacDonald, G. M., \& Gallego-Sala, A. (2014). Peatland
initiation, carbon accumulation, and 2 ka depth in the james bay lowland
and adjacent regions. \emph{Arctic, Antarctic, and Alpine Research},
\emph{46}(1), 19--39. \url{https://doi.org/10.1657/1938-4246-46.1.19}

\leavevmode\hypertarget{ref-Hood_2009}{}%
Hood, E., Fellman, J., Spencer, R. G. M., Hernes, P. J., Edwards, R.,
D'Amore, D., \& Scott, D. (2009). Glaciers as a source of ancient and
labile organic matter to the marine environment. \emph{Nature},
\emph{462}(7276), 1044--1047. \url{https://doi.org/10.1038/nature08580}

\leavevmode\hypertarget{ref-Hope_2005}{}%
Hope, G., Chokkalingam, U., \& Anwar, S. (2005). The stratigraphy and
fire history of the kutai peatlands, kalimantan, indonesia.
\emph{Quaternary Research}, \emph{64}(3), 407--417.
\url{https://doi.org/10.1016/j.yqres.2005.08.009}

\leavevmode\hypertarget{ref-Horwath_2008}{}%
Horwath, J. L., Sletten, R. S., Hagedorn, B., \& Hallet, B. (2008).
Spatial and temporal distribution of soil organic carbon in nonsorted
striped patterned ground of the high arctic. \emph{Journal of
Geophysical Research}, \emph{113}(G3).
\url{https://doi.org/10.1029/2007jg000511}

\leavevmode\hypertarget{ref-Hribljan_2016}{}%
Hribljan, J. A., Suárez, E., Heckman, K. A., Lilleskov, E. A., \&
Chimner, R. A. (2016). Peatland carbon stocks and accumulation rates in
the ecuadorian páramo. \emph{Wetlands Ecology and Management},
\emph{24}(2), 113--127. \url{https://doi.org/10.1007/s11273-016-9482-2}

\leavevmode\hypertarget{ref-Hsieh_1996}{}%
Hsieh, Y.-P. (1996). Soil organic carbon pools of two tropical soils
inferred by carbon signatures. \emph{Soil Science Society of America
Journal}, \emph{60}(4), 1117--1121.
\url{https://doi.org/10.2136/sssaj1996.03615995006000040022x}

\leavevmode\hypertarget{ref-Huang_1996}{}%
Huang, Y., Bol, R., Harkness, D. D., Ineson, P., \& Eglinton, G. (1996).
Post-glacial variations in distributions, 13C and 14C contents of
aliphatic hydrocarbons and bulk organic matter in three types of british
acid upland soils. \emph{Organic Geochemistry}, \emph{24}(3), 273--287.
\url{https://doi.org/10.1016/0146-6380(96)00039-3}

\leavevmode\hypertarget{ref-Huang_1999}{}%
Huang, Y., Li, B., Bryant, C., Bol, R., \& Eglinton, G. (1999).
Radiocarbon dating of aliphatic hydrocarbons a new approach for dating
passive-fraction carbon in soil horizons. \emph{Soil Science Society of
America Journal}, \emph{63}(5), 1181--1187.
\url{https://doi.org/10.2136/sssaj1999.6351181x}

\leavevmode\hypertarget{ref-Hugelius_2012}{}%
Hugelius, G., Routh, J., Kuhry, P., \& Crill, P. (2012). Mapping the
degree of decomposition and thaw remobilization potential of soil
organic matter in discontinuous permafrost terrain. \emph{Journal of
Geophysical Research: Biogeosciences}, \emph{117}(G2), n/a--n/a.
\url{https://doi.org/10.1029/2011jg001873}

\leavevmode\hypertarget{ref-Hunt_2013}{}%
Hunt, S., Yu, Z., \& Jones, M. (2013). Lateglacial and holocene climate,
disturbance and permafrost peatland dynamics on the seward peninsula,
western alaska. \emph{Quaternary Science Reviews}, \emph{63}, 42--58.
\url{https://doi.org/10.1016/j.quascirev.2012.11.019}

\leavevmode\hypertarget{ref-1990}{}%
International conference on soils and the greenhouse effect. (1990).
\emph{COSPAR Information Bulletin}, \emph{1990}(118), 6--7.
\url{https://doi.org/10.1016/0045-8732(90)90051-o}

\leavevmode\hypertarget{ref-James_2019}{}%
James, J. N., Gross, C. D., Dwivedi, P., Myers, T., Santos, F.,
Bernardi, R., Faria, M. F. de, Guerrini, I. A., Harrison, R., \& Butman,
D. (2019). Land use change alters the radiocarbon age and composition of
soil and water-soluble organic matter in the brazilian cerrado.
\emph{Geoderma}, \emph{345}, 38--50.
\url{https://doi.org/10.1016/j.geoderma.2019.03.019}

\leavevmode\hypertarget{ref-JANKOVSK__2006}{}%
JANKOVSKÁ, V., ANDREEV, A. A., \& PANOVA, N. K. (2006). Holocene
environmental history on the eastern slope of the polar ural mountains,
russia. \emph{Boreas}, \emph{35}(4), 650--661.
\url{https://doi.org/10.1111/j.1502-3885.2006.tb01171.x}

\leavevmode\hypertarget{ref-Jasinski_1998}{}%
Jasinski, J., Warner, B. G., Andreev, A. A., Aravena, R., Gilbert, S.
E., Zeeb, B. A., Smol, J. P., \& Velichko, A. A. (1998). Holocene
environmental history of a peatland in the lena river valley, siberia.
\emph{Canadian Journal of Earth Sciences}, \emph{35}(6), 637--648.
\url{https://doi.org/10.1139/e98-015}

\leavevmode\hypertarget{ref-Johnston_2014}{}%
Johnston, C. E., Ewing, S. A., Harden, J. W., Varner, R. K., Wickland,
K. P., Koch, J. C., Fuller, C. C., Manies, K., \& Jorgenson, M. T.
(2014). Effect of permafrost thaw on CO 2 and CH 4 exchange in a western
alaska peatland chronosequence. \emph{Environmental Research Letters},
\emph{9}(8), 085004. \url{https://doi.org/10.1088/1748-9326/9/8/085004}

\leavevmode\hypertarget{ref-Jones_2012}{}%
Jones, M. C., Grosse, G., Jones, B. M., \& Anthony, K. W. (2012). Peat
accumulation in drained thermokarst lake basins in continuous, ice-rich
permafrost, northern seward peninsula, alaska. \emph{Journal of
Geophysical Research: Biogeosciences}, \emph{117}(G2), n/a--n/a.
\url{https://doi.org/10.1029/2011jg001766}

\leavevmode\hypertarget{ref-Jones_2009}{}%
Jones, M. C., Peteet, D. M., Kurdyla, D., \& Guilderson, T. (2009).
Climate and vegetation history from a 14,000-year peatland record, kenai
peninsula, alaska. \emph{Quaternary Research}, \emph{72}(2), 207--217.
\url{https://doi.org/10.1016/j.yqres.2009.04.002}

\leavevmode\hypertarget{ref-Jones_2014}{}%
Jones, M. C., Wooller, M., \& Peteet, D. M. (2014). A deglacial and
holocene record of climate variability in south-central alaska from
stable oxygen isotopes and plant macrofossils in peat. \emph{Quaternary
Science Reviews}, \emph{87}, 1--11.
\url{https://doi.org/10.1016/j.quascirev.2013.12.025}

\leavevmode\hypertarget{ref-Kaiser_2007}{}%
Kaiser, C., Meyer, H., Biasi, C., Rusalimova, O., Barsukov, P., \&
Richter, A. (2007). Conservation of soil organic matter through
cryoturbation in arctic soils in siberia. \emph{Journal of Geophysical
Research}, \emph{112}(G2). \url{https://doi.org/10.1029/2006jg000258}

\leavevmode\hypertarget{ref-Karhu_2010}{}%
Karhu, K., Fritze, H., Hämäläinen, K., Vanhala, P., Jungner, H.,
Oinonen, M., Sonninen, E., Tuomi, M., Spetz, P., Kitunen, V., \& Liski,
J. (2010). Temperature sensitivity of soil carbon fractions in boreal
forest soil. \emph{Ecology}, \emph{91}(2), 370--376.
\url{https://doi.org/10.1890/09-0478.1}

\leavevmode\hypertarget{ref-Katsuno_2010}{}%
Katsuno, K., Miyairi, Y., Tamura, K., Matsuzaki, H., \& Fukuda, K.
(2010). A study of the carbon dynamics of japanese grassland and forest
using 14C and 13C. \emph{Nuclear Instruments and Methods in Physics
Research Section B: Beam Interactions with Materials and Atoms},
\emph{268}(7-8), 1106--1109.
\url{https://doi.org/10.1016/j.nimb.2009.10.110}

\leavevmode\hypertarget{ref-Kelly_2020}{}%
Kelly, T. J., Lawson, I. T., Roucoux, K. H., Baker, T. R., \& Coronado,
E. N. H. (2020). Patterns and drivers of development in a west amazonian
peatland during the late holocene. \emph{Quaternary Science Reviews},
\emph{230}, 106168.
\url{https://doi.org/10.1016/j.quascirev.2020.106168}

\leavevmode\hypertarget{ref-Kelly_2018}{}%
Kelly, T. J., Lawson, I. T., Roucoux, K. H., Baker, T. R.,
Honorio-Coronado, E. N., Jones, T. D., \& Panduro, S. R. (2018).
Continuous human presence without extensive reductions in forest cover
over the past 2500 years in an aseasonal amazonian rainforest.
\emph{Journal of Quaternary Science}, \emph{33}(4), 369--379.
\url{https://doi.org/10.1002/jqs.3019}

\leavevmode\hypertarget{ref-Kelly_2017}{}%
Kelly, T. J., Lawson, I. T., Roucoux, K. H., Baker, T. R., Jones, T. D.,
\& Sanderson, N. K. (2017). The vegetation history of an amazonian domed
peatland. \emph{Palaeogeography, Palaeoclimatology, Palaeoecology},
\emph{468}, 129--141. \url{https://doi.org/10.1016/j.palaeo.2016.11.039}

\leavevmode\hypertarget{ref-Kettles_2000}{}%
Kettles, I. M., Garneau, M., \& Jetté, H. (2000). \emph{Macrofossil,
pollen, and geochemical records of peatlands in the knosheo lake and
detour lake areas, northern ontario}. Natural Resources
Canada/CMSS/Information Management. \url{https://doi.org/10.4095/211326}

\leavevmode\hypertarget{ref-Kettles_2003}{}%
Kettles, I. M., Robinson, S. D., Bastien, D. -F, Garneau, M., \& Hall,
G. E. M. (2003). \emph{Physical, geochemical, macrofossil, and ground
penetrating radar information on fourteen permafrost-affected peatlands
in the mackenzie valley, northwest territories}. Natural Resources
Canada/CMSS/Information Management. \url{https://doi.org/10.4095/214221}

\leavevmode\hypertarget{ref-Khomo_2017}{}%
Khomo, L., Trumbore, S., Bern, C. R., \& Chadwick, O. A. (2017).
Timescales of carbon turnover in soils with mixed crystalline
mineralogies. \emph{SOIL}, \emph{3}(1), 17--30.
\url{https://doi.org/10.5194/soil-3-17-2017}

\leavevmode\hypertarget{ref-Klapstein_2014}{}%
Klapstein, S. J., Turetsky, M. R., McGuire, A. D., Harden, J. W.,
Czimczik, C. I., Xu, X., Chanton, J. P., \& Waddington, J. M. (2014).
Controls on methane released through ebullition in peatlands affected by
permafrost degradation. \emph{Journal of Geophysical Research:
Biogeosciences}, \emph{119}(3), 418--431.
\url{https://doi.org/10.1002/2013jg002441}

\leavevmode\hypertarget{ref-Kleber_2005}{}%
Kleber, M., Mikutta, R., Torn, M. S., \& Jahn, R. (2005). Poorly
crystalline mineral phases protect organic matter in acid subsoil
horizons. \emph{European Journal of Soil Science}, \emph{0}(0),
050912034650054. \url{https://doi.org/10.1111/j.1365-2389.2005.00706.x}

\leavevmode\hypertarget{ref-KLEBER_2011}{}%
KLEBER, M., NICO, P. S., PLANTE, A., FILLEY, T., KRAMER, M., SWANSTON,
C., \& SOLLINS, P. (2011). Old and stable soil organic matter is not
necessarily chemically recalcitrant: Implications for modeling concepts
and temperature sensitivity. \emph{Global Change Biology}, \emph{17}(2),
1097--1107. \url{https://doi.org/10.1111/j.1365-2486.2010.02278.x}

\leavevmode\hypertarget{ref-Klein_2013}{}%
Klein, E. S., Yu, Z., \& Booth, R. K. (2013). Recent increase in
peatland carbon accumulation in a thermokarst lake basin in southwestern
alaska. \emph{Palaeogeography, Palaeoclimatology, Palaeoecology},
\emph{392}, 186--195. \url{https://doi.org/10.1016/j.palaeo.2013.09.009}

\leavevmode\hypertarget{ref-de_Klerk_2011}{}%
Klerk, P. de, Donner, N., Karpov, N. S., Minke, M., \& Joosten, H.
(2011). Short-term dynamics of a low-centred ice-wedge polygon near
chokurdakh (NE yakutia, NE siberia) and climate change during the last
ca 1250 years. \emph{Quaternary Science Reviews}, \emph{30}(21-22),
3013--3031. \url{https://doi.org/10.1016/j.quascirev.2011.06.016}

\leavevmode\hypertarget{ref-KOARASHI_2009}{}%
KOARASHI, J., ATARASHI-ANDOH, M., ISHIZUKA, S., MIURA, S., SAITO, T., \&
HIRAI, K. (2009). Quantitative aspects of heterogeneity in soil organic
matter dynamics in a cool-temperate japanese beech forest: A
radiocarbon-based approach. \emph{Global Change Biology}, \emph{15}(3),
631--642. \url{https://doi.org/10.1111/j.1365-2486.2008.01745.x}

\leavevmode\hypertarget{ref-Koarashi_2012}{}%
Koarashi, J., Hockaday, W. C., Masiello, C. A., \& Trumbore, S. E.
(2012). Dynamics of decadally cycling carbon in subsurface soils.
\emph{Journal of Geophysical Research: Biogeosciences}, \emph{117}(G3),
n/a--n/a. \url{https://doi.org/10.1029/2012jg002034}

\leavevmode\hypertarget{ref-Koarashi_2005}{}%
Koarashi, J., Iida, T., \& Asano, T. (2005). Radiocarbon and stable
carbon isotope compositions of chemically fractionated soil organic
matter in a temperate-zone forest. \emph{Journal of Environmental
Radioactivity}, \emph{79}(2), 137--156.
\url{https://doi.org/10.1016/j.jenvrad.2004.06.002}

\leavevmode\hypertarget{ref-Kokfelt_2010}{}%
Kokfelt, U., Reuss, N., Struyf, E., Sonesson, M., Rundgren, M., Skog,
G., Rosén, P., \& Hammarlund, D. (2010). Wetland development, permafrost
history and nutrient cycling inferred from late holocene peat and lake
sediment records in subarctic sweden. \emph{Journal of Paleolimnology},
\emph{44}(1), 327--342. \url{https://doi.org/10.1007/s10933-010-9406-8}

\leavevmode\hypertarget{ref-Kondo_2010}{}%
Kondo, M., Uchida, M., \& Shibata, Y. (2010). Radiocarbon-based
residence time estimates of soil organic carbon in a temperate forest:
Case study for the density fractionation for japanese volcanic ash soil.
\emph{Nuclear Instruments and Methods in Physics Research Section B:
Beam Interactions with Materials and Atoms}, \emph{268}(7-8),
1073--1076. \url{https://doi.org/10.1016/j.nimb.2009.10.101}

\leavevmode\hypertarget{ref-Kovda_2001}{}%
Kovda, I., Lynn, W., Williams, D., \& Chichagova, O. (2001). Radiocarbon
age of vertisols and its interpretation using data on gilgai complex in
the north caucasus. \emph{Radiocarbon}, \emph{43}(2B), 603--610.
\url{https://doi.org/10.1017/s0033822200041254}

\leavevmode\hypertarget{ref-K_gel_Knabner_2008}{}%
Kögel-Knabner, I., Guggenberger, G., Kleber, M., Kandeler, E., Kalbitz,
K., Scheu, S., Eusterhues, K., \& Leinweber, P. (2008). Organo-mineral
associations in temperate soils: Integrating biology, mineralogy, and
organic matter chemistry. \emph{Journal of Plant Nutrition and Soil
Science}, \emph{171}(1), 61--82.
\url{https://doi.org/10.1002/jpln.200700048}

\leavevmode\hypertarget{ref-Kramer_2016}{}%
Kramer, M. G., \& Chadwick, O. A. (2016). Controls on carbon storage and
weathering in volcanic soils across a high-elevation climate gradient on
mauna kea, hawaii. \emph{Ecology}, \emph{97}(9), 2384--2395.
\url{https://doi.org/10.1002/ecy.1467}

\leavevmode\hypertarget{ref-Kramer_2012}{}%
Kramer, M. G., Sanderman, J., Chadwick, O. A., Chorover, J., \&
Vitousek, P. M. (2012). Long-term carbon storage through retention of
dissolved aromatic acids by reactive particles in soil. \emph{Global
Change Biology}, \emph{18}(8), 2594--2605.
\url{https://doi.org/10.1111/j.1365-2486.2012.02681.x}

\leavevmode\hypertarget{ref-KREMENETSKI_2008}{}%
KREMENETSKI, C., VASCHALOVA, T., GORIACHKIN, S., CHERKINSKY, A., \&
SULERZHITSKY, L. (2008). Holocene pollen stratigraphy and bog
development in the western part of the kola peninsula, russia.
\emph{Boreas}, \emph{26}(2), 91--102.
\url{https://doi.org/10.1111/j.1502-3885.1997.tb00656.x}

\leavevmode\hypertarget{ref-Krull_2006}{}%
Krull, E. S., Bestland, E. A., Skjemstad, J. O., \& Parr, J. F. (2006).
Geochemistry (\(\updelta\)13C, \(\updelta\)15N, 13C NMR) and residence
times (14C and OSL) of soil organic matter from red-brown earths of
south australia: Implications for soil genesis. \emph{Geoderma},
\emph{132}(3-4), 344--360.
\url{https://doi.org/10.1016/j.geoderma.2005.06.001}

\leavevmode\hypertarget{ref-Krull_2003}{}%
Krull, E. S., \& Skjemstad, J. O. (2003). \(\updelta\)13C and
\(\updelta\)15N profiles in 14C-dated oxisol and vertisols as a function
of soil chemistry and mineralogy. \emph{Geoderma}, \emph{112}(1-2),
1--29. \url{https://doi.org/10.1016/s0016-7061(02)00291-4}

\leavevmode\hypertarget{ref-Krull_2005}{}%
Krull, E. S., Skjemstad, J. O., Burrows, W. H., Bray, S. G., Wynn, J.
G., Bol, R., Spouncer, L., \& Harms, B. (2005). Recent vegetation
changes in central queensland, australia: Evidence from \(\updelta\)13C
and 14C analyses of soil organic matter. \emph{Geoderma},
\emph{126}(3-4), 241--259.
\url{https://doi.org/10.1016/j.geoderma.2004.09.012}

\leavevmode\hypertarget{ref-https:ux2fux2fdoi.orgux2f10.5281ux2fzenodo.2645510}{}%
Kuhnen, Á., Matschullat, J., Sierra, C. A., \& Lima, R. M. B. de.
(2019). \emph{C-isotopic signatures and soil properties of amazon basin
oxisols}. Zenodo. \url{https://doi.org/10.5281/ZENODO.2645510}

\leavevmode\hypertarget{ref-Kuhry_1997}{}%
Kuhry, P. (1997). The palaeoecology of a treed bog in western boreal
canada: A study based on microfossils, macrofossils and physico-chemical
properties. \emph{Review of Palaeobotany and Palynology},
\emph{96}(1-2), 183--224.
\url{https://doi.org/10.1016/s0034-6667(96)00018-8}

\leavevmode\hypertarget{ref-KUHRY_2008}{}%
KUHRY, P. (2008). Palsa and peat plateau development in the hudson bay
lowlands, canada: Timing, pathways and causes. \emph{Boreas},
\emph{37}(2), 316--327.
\url{https://doi.org/10.1111/j.1502-3885.2007.00022.x}

\leavevmode\hypertarget{ref-Kuhry_1996}{}%
Kuhry, P., \& Vitt, D. H. (1996). Fossil carbon/nitrogen ratios as a
measure of peat decomposition. \emph{Ecology}, \emph{77}(1), 271--275.
\url{https://doi.org/10.2307/2265676}

\leavevmode\hypertarget{ref-Kultti_2004}{}%
Kultti, S., Oksanen, P., \& Väliranta, M. (2004). Holocene tree line,
permafrost, and climate dynamics in the nenets region, east european
arctic. \emph{Canadian Journal of Earth Sciences}, \emph{41}(10),
1141--1158. \url{https://doi.org/10.1139/e04-058}

\leavevmode\hypertarget{ref-Kwon_2019}{}%
Kwon, M. J., Natali, S. M., Pries, C. E. H., Schuur, E. A. G., Steinhof,
A., Crummer, K. G., Zimov, N., Zimov, S. A., Heimann, M., Kolle, O., \&
Göckede, M. (2019). Drainage enhances modern soil carbon contribution
but reduces old soil carbon contribution to ecosystem respiration in
tundra ecosystems. \emph{Global Change Biology}, \emph{25}(4),
1315--1325. \url{https://doi.org/10.1111/gcb.14578}

\leavevmode\hypertarget{ref-Ladyman_1980}{}%
Ladyman, S. J., \& Harkness, D. D. (1980). Carbon isotope measurement as
an index of soil development. \emph{Radiocarbon}, \emph{22}(3),
885--891. \url{https://doi.org/10.1017/s0033822200010286}

\leavevmode\hypertarget{ref-Lamarre_2012}{}%
Lamarre, A., Garneau, M., \& Asnong, H. (2012). Holocene
paleohydrological reconstruction and carbon accumulation of a permafrost
peatland using testate amoeba and macrofossil analyses, kuujjuarapik,
subarctic québec, canada. \emph{Review of Palaeobotany and Palynology},
\emph{186}, 131--141.
\url{https://doi.org/10.1016/j.revpalbo.2012.04.009}

\leavevmode\hypertarget{ref-Laskar_2012}{}%
Laskar, A. H., Yadava, M. G., \& Ramesh, R. (2012). Radiocarbon and
stable carbon isotopes in two soil profiles from northeast india.
\emph{Radiocarbon}, \emph{54}(1), 81--89.
\url{https://doi.org/10.2458/azu_js_rc.v54i1.15840}

\leavevmode\hypertarget{ref-Lassey_1996}{}%
Lassey, K. R., Tate, K. R., Sparks, R. J., \& Claydon, J. J. (1996).
Historic measurements of radiocarbon in new zealand soils.
\emph{Radiocarbon}, \emph{38}(2), 253--270.
\url{https://doi.org/10.1017/s003382220001763x}

\leavevmode\hypertarget{ref-Lavoie_1995}{}%
Lavoie, C., \& Payette, S. (1995). Analyse macrofossile dune palse
subarctique (québec nordique). \emph{Canadian Journal of Botany},
\emph{73}(4), 527--537. \url{https://doi.org/10.1139/b95-054}

\leavevmode\hypertarget{ref-Lavoie_2011}{}%
Lavoie, M., Mack, M. C., \& Schuur, E. A. G. (2011). Effects of elevated
nitrogen and temperature on carbon and nitrogen dynamics in alaskan
arctic and boreal soils. \emph{Journal of Geophysical Research},
\emph{116}(G3). \url{https://doi.org/10.1029/2010jg001629}

\leavevmode\hypertarget{ref-Lawrence_2015}{}%
Lawrence, C. R., Harden, J. W., Xu, X., Schulz, M. S., \& Trumbore, S.
E. (2015). Long-term controls on soil organic carbon with depth and
time: A case study from the cowlitz river chronosequence, WA USA.
\emph{Geoderma}, \emph{247-248}, 73--87.
\url{https://doi.org/10.1016/j.geoderma.2015.02.005}

\leavevmode\hypertarget{ref-Lawrence_2021}{}%
Lawrence, C. R., Schulz, M. S., Masiello, C. A., Chadwick, O. A., \&
Harden, J. W. (2021). The trajectory of soil development and its
relationship to soil carbon dynamics. \emph{Geoderma}, \emph{403},
115378. \url{https://doi.org/10.1016/j.geoderma.2021.115378}

\leavevmode\hypertarget{ref-L_hteenoja_2011}{}%
Lähteenoja, O., Reátegui, Y. R., Räsänen, M., Torres, D. D. C., Oinonen,
M., \& Page, S. (2011). The large amazonian peatland carbon sink in the
subsiding pastaza-marañón foreland basin, peru. \emph{Global Change
Biology}, \emph{18}(1), 164--178.
\url{https://doi.org/10.1111/j.1365-2486.2011.02504.x}

\leavevmode\hypertarget{ref-L_HTEENOJA_2009}{}%
LÄHTEENOJA, O., RUOKOLAINEN, K., SCHULMAN, L., \& OINONEN, M. (2009).
Amazonian peatlands: An ignored c sink and potential source.
\emph{Global Change Biology}, \emph{15}(9), 2311--2320.
\url{https://doi.org/10.1111/j.1365-2486.2009.01920.x}

\leavevmode\hypertarget{ref-Leavitt_2007}{}%
Leavitt, S., Follett, R., Kimble, J., \& Pruessner, E. (2007).
Radiocarbon and \(\updelta\)13C depth profiles of soil organic carbon in
the u.s. Great plains: A possible spatial record of paleoenvironment and
paleovegetation. \emph{Quaternary International}, \emph{162-163},
21--34. \url{https://doi.org/10.1016/j.quaint.2006.10.033}

\leavevmode\hypertarget{ref-Ledru_2001}{}%
Ledru, M.-P. (2001). Late holocene rainforest disturbance in french
guiana. \emph{Review of Palaeobotany and Palynology}, \emph{115}(3-4),
161--170. \url{https://doi.org/10.1016/s0034-6667(01)00068-9}

\leavevmode\hypertarget{ref-Lee_2011}{}%
Lee, H., Schuur, E. A. G., Inglett, K. S., Lavoie, M., \& Chanton, J. P.
(2011). The rate of permafrost carbon release under aerobic and
anaerobic conditions and its potential effects on climate. \emph{Global
Change Biology}, \emph{18}(2), 515--527.
\url{https://doi.org/10.1111/j.1365-2486.2011.02519.x}

\leavevmode\hypertarget{ref-LEIFELD_2009}{}%
LEIFELD, J., ZIMMERMANN, M., FUHRER, J., \& CONEN, F. (2009). Storage
and turnover of carbon in grassland soils along an elevation gradient in
the swiss alps. \emph{Global Change Biology}, \emph{15}(3), 668--679.
\url{https://doi.org/10.1111/j.1365-2486.2008.01782.x}

\leavevmode\hypertarget{ref-Leith_2014}{}%
Leith, F. I., Garnett, M. H., Dinsmore, K. J., Billett, M. F., \& Heal,
K. V. (2014). Source and age of dissolved and gaseous carbon in a
peatlandriparianstream continuum: A dual isotope (14C and
\(\updelta\)13C) analysis. \emph{Biogeochemistry}, \emph{119}(1-3),
415--433. \url{https://doi.org/10.1007/s10533-014-9977-y}

\leavevmode\hypertarget{ref-Li_2010}{}%
Li, Y., \& Mathews, B. W. (2010). Effect of conversion of sugarcane
plantation to forest and pasture on soil carbon in hawaii. \emph{Plant
and Soil}, \emph{335}(1-2), 245--253.
\url{https://doi.org/10.1007/s11104-010-0412-4}

\leavevmode\hypertarget{ref-Liu_2006}{}%
Liu, W., Moriizumi, J., Yamazawa, H., \& Iida, T. (2006). Depth profiles
of radiocarbon and carbon isotopic compositions of organic matter and
CO2 in a forest soil. \emph{Journal of Environmental Radioactivity},
\emph{90}(3), 210--223.
\url{https://doi.org/10.1016/j.jenvrad.2006.07.003}

\leavevmode\hypertarget{ref-Loisel_2010}{}%
Loisel, J., \& Garneau, M. (2010). Late holocene paleoecohydrology and
carbon accumulation estimates from two boreal peat bogs in eastern
canada: Potential and limits of multi-proxy archives.
\emph{Palaeogeography, Palaeoclimatology, Palaeoecology},
\emph{291}(3-4), 493--533.
\url{https://doi.org/10.1016/j.palaeo.2010.03.020}

\leavevmode\hypertarget{ref-Loisel_2014}{}%
Loisel, J., Yu, Z., Beilman, D. W., Camill, P., Alm, J., Amesbury, M.
J., Anderson, D., Andersson, S., Bochicchio, C., Barber, K., Belyea, L.
R., Bunbury, J., Chambers, F. M., Charman, D. J., Vleeschouwer, F. D.,
Fiałkiewicz-Kozieł, B., Finkelstein, S. A., Gałka, M., Garneau, M.,
\ldots{} Zhou, W. (2014). A database and synthesis of northern peatland
soil properties and holocene carbon and nitrogen accumulation. \emph{The
Holocene}, \emph{24}(9), 1028--1042.
\url{https://doi.org/10.1177/0959683614538073}

\leavevmode\hypertarget{ref-Lupascu_2020}{}%
Lupascu, M., Akhtar, H., Smith, T. E. L., \& Sukri, R. S. (2020).
Post-fire carbon dynamics in the tropical peat swamp forests of brunei
reveal long-term elevated CH 4 flux. \emph{Global Change Biology},
\emph{26}(9), 5125--5145. \url{https://doi.org/10.1111/gcb.15195}

\leavevmode\hypertarget{ref-Lupascu_2018}{}%
Lupascu, M., Czimczik, C. I., Welker, M. C., Ziolkowski, L. A., Cooper,
E. J., \& Welker, J. M. (2018). Winter ecosystem respiration and sources
of CO 2 from the high arctic tundra of svalbard: Response to a deeper
snow experiment. \emph{Journal of Geophysical Research: Biogeosciences},
\emph{123}(8), 2627--2642. \url{https://doi.org/10.1029/2018jg004396}

\leavevmode\hypertarget{ref-Lupascu_2013}{}%
Lupascu, M., Welker, J. M., Seibt, U., Maseyk, K., Xu, X., \& Czimczik,
C. I. (2013). High arctic wetting reduces permafrost carbon feedbacks to
climate warming. \emph{Nature Climate Change}, \emph{4}(1), 51--55.
\url{https://doi.org/10.1038/nclimate2058}

\leavevmode\hypertarget{ref-Lybrand_2017}{}%
Lybrand, R. A., Heckman, K., \& Rasmussen, C. (2017). Soil organic
carbon partitioning and \(\upDelta\)14C variation in desert and conifer
ecosystems of southern arizona. \emph{Biogeochemistry}, \emph{134}(3),
261--277. \url{https://doi.org/10.1007/s10533-017-0360-7}

\leavevmode\hypertarget{ref-Mann_2015}{}%
Mann, P. J., Eglinton, T. I., McIntyre, C. P., Zimov, N., Davydova, A.,
Vonk, J. E., Holmes, R. M., \& Spencer, R. G. M. (2015). Utilization of
ancient permafrost carbon in headwaters of arctic fluvial networks.
\emph{Nature Communications}, \emph{6}(1).
\url{https://doi.org/10.1038/ncomms8856}

\leavevmode\hypertarget{ref-Marin_Spiotta_2011}{}%
Marin-Spiotta, E., Chadwick, O. A., Kramer, M., \& Carbone, M. S.
(2011). Carbon delivery to deep mineral horizons in hawaiian rain forest
soils. \emph{Journal of Geophysical Research}, \emph{116}(G3).
\url{https://doi.org/10.1029/2010jg001587}

\leavevmode\hypertarget{ref-Mar_n_Spiotta_2008}{}%
Marı'n-Spiotta, E., Swanston, C. W., Torn, M. S., Silver, W. L., \&
Burton, S. D. (2008). Chemical and mineral control of soil carbon
turnover in abandoned tropical pastures. \emph{Geoderma},
\emph{143}(1-2), 49--62.
\url{https://doi.org/10.1016/j.geoderma.2007.10.001}

\leavevmode\hypertarget{ref-Mariotti_1994}{}%
Mariotti, A., \& Peterschmitt, E. (1994). Forest savanna ecotone
dynamics in india as revealed by carbon isotope ratios of soil organic
matter. \emph{Oecologia}, \emph{97}(4), 475--480.
\url{https://doi.org/10.1007/bf00325885}

\leavevmode\hypertarget{ref-Martel_1974}{}%
Martel, Y. A., \& Paul, E. A. (1974). The use of radiocarbon dating of
organic matter in the study of soil genesis. \emph{Soil Science Society
of America Journal}, \emph{38}(3), 501--506.
\url{https://doi.org/10.2136/sssaj1974.03615995003800030033x}

\leavevmode\hypertarget{ref-Martens_1992}{}%
Martens, C. S., Kelley, C. A., Chanton, J. P., \& Showers, W. J. (1992).
Carbon and hydrogen isotopic characterization of methane from wetlands
and lakes of the yukon-kuskokwim delta, western alaska. \emph{Journal of
Geophysical Research}, \emph{97}(D15), 16689.
\url{https://doi.org/10.1029/91jd02885}

\leavevmode\hypertarget{ref-Martinelli_1996}{}%
Martinelli, I. A., Pessenda, L. C. R., Espinoza, E., Camargo, P. B.,
Telles, F. C., Cerri, C. C., Victoria, R. L., Aravena, R., Richey, J.,
\& Trumbore, S. (1996). Carbon-13 variation with depth in soils of
brazil and climate change during the quaternary. \emph{Oecologia},
\emph{106}(3), 376--381. \url{https://doi.org/10.1007/bf00334565}

\leavevmode\hypertarget{ref-Masiello_2004}{}%
Masiello, C. A., Chadwick, O. A., Southon, J., Torn, M. S., \& Harden,
J. W. (2004). Weathering controls on mechanisms of carbon storage in
grassland soils. \emph{Global Biogeochemical Cycles}, \emph{18}(4),
n/a--n/a. \url{https://doi.org/10.1029/2004gb002219}

\leavevmode\hypertarget{ref-Massa_2021}{}%
Massa, C., Beilman, D. W., Nichols, J. E., \& Timm, O. E. (2021).
Central pacific hydroclimate over the last 45,000 years:
Molecular-isotopic evidence from leaf wax in a hawaiʻi peatland.
\emph{Quaternary Science Reviews}, \emph{253}, 106744.
\url{https://doi.org/10.1016/j.quascirev.2020.106744}

\leavevmode\hypertarget{ref-Mayer_2018}{}%
Mayer, S., Schwindt, D., Steffens, M., Völkel, J., \& Kögel-Knabner, I.
(2018). Drivers of organic carbon allocation in a temperate
slope-floodplain catena under agricultural use. \emph{Geoderma},
\emph{327}, 63--72. \url{https://doi.org/10.1016/j.geoderma.2018.04.021}

\leavevmode\hypertarget{ref-McClaran_2000}{}%
McClaran, M. P., \& Umlauf, M. (2000). Desert grassland dynamics
estimated from carbon isotopes in grass phytoliths and soil organic
matter. \emph{Journal of Vegetation Science}, \emph{11}(1), 71--76.
\url{https://doi.org/10.2307/3236777}

\leavevmode\hypertarget{ref-McFarlane_2018}{}%
McFarlane, K. J., Hanson, P. J., Iversen, C. M., Phillips, J. R., \&
Brice, D. J. (2018). Local spatial heterogeneity of holocene carbon
accumulation throughout the peat profile of an ombrotrophic northern
minnesota bog. \emph{Radiocarbon}, \emph{60}(3), 941--962.
\url{https://doi.org/10.1017/rdc.2018.37}

\leavevmode\hypertarget{ref-McFarlane_2012}{}%
McFarlane, K. J., Torn, M. S., Hanson, P. J., Porras, R. C., Swanston,
C. W., Callaham, M. A., \& Guilderson, T. P. (2012). Comparison of soil
organic matter dynamics at five temperate deciduous forests with
physical fractionation and radiocarbon measurements.
\emph{Biogeochemistry}, \emph{112}(1-3), 457--476.
\url{https://doi.org/10.1007/s10533-012-9740-1}

\leavevmode\hypertarget{ref-Mergelov_2020}{}%
Mergelov, N., Dolgikh, A., Shorkunov, I., Zazovskaya, E., Soina, V.,
Yakushev, A., Fedorov-Davydov, D., Pryakhin, S., \& Dobryansky, A.
(2020). Hypolithic communities shape soils and organic matter reservoirs
in the ice-free landscapes of east antarctica. \emph{Scientific
Reports}, \emph{10}(1). \url{https://doi.org/10.1038/s41598-020-67248-3}

\leavevmode\hypertarget{ref-Meyer_2012}{}%
Meyer, S., Leifeld, J., Bahn, M., \& Fuhrer, J. (2012). Free and
protected soil organic carbon dynamics respond differently to
abandonment of mountain grassland. \emph{Biogeosciences}, \emph{9}(2),
853--865. \url{https://doi.org/10.5194/bg-9-853-2012}

\leavevmode\hypertarget{ref-Mikutta_2009}{}%
Mikutta, R., Schaumann, G. E., Gildemeister, D., Bonneville, S., Kramer,
M. G., Chorover, J., Chadwick, O. A., \& Guggenberger, G. (2009).
Biogeochemistry of mineralorganic associations across a long-term
mineralogical soil gradient (0.34100kyr), hawaiian islands.
\emph{Geochimica et Cosmochimica Acta}, \emph{73}(7), 2034--2060.
\url{https://doi.org/10.1016/j.gca.2008.12.028}

\leavevmode\hypertarget{ref-Milton_1997}{}%
Milton, G. M., \& Kramer, S. J. (1997). Using 14C as a tracer of carbon
accumulation and turnover in soils. \emph{Radiocarbon}, \emph{40}(2),
999--1011. \url{https://doi.org/10.1017/s003382220001897x}

\leavevmode\hypertarget{ref-Monreal_1997}{}%
Monreal, C. M., Schulten, H.-R., \& Kodama, H. (1997). Age, turnover and
molecular diversity of soil organic matter in aggregates of a gleysol.
\emph{Canadian Journal of Soil Science}, \emph{77}(3), 379--388.
\url{https://doi.org/10.4141/s95-064}

\leavevmode\hypertarget{ref-van_Mourik_2010}{}%
Mourik, J. van, Nierop, K., \& Vandenberghe, D. (2010). Radiocarbon and
optically stimulated luminescence dating based chronology of a
polycyclic driftsand sequence at weerterbergen (SE netherlands).
\emph{CATENA}, \emph{80}(3), 170--181.
\url{https://doi.org/10.1016/j.catena.2009.11.004}

\leavevmode\hypertarget{ref-Mueller_2014}{}%
Mueller, C. W., Gutsch, M., Kothieringer, K., Leifeld, J., Rethemeyer,
J., Brueggemann, N., \& Kögel-Knabner, I. (2014). Bioavailability and
isotopic composition of CO2 released from incubated soil organic matter
fractions. \emph{Soil Biology and Biochemistry}, \emph{69}, 168--178.
\url{https://doi.org/10.1016/j.soilbio.2013.11.006}

\leavevmode\hypertarget{ref-Muhr_2009}{}%
Muhr, J., \& Borken, W. (2009). Delayed recovery of soil respiration
after wetting of dry soil further reduces c losses from a norway spruce
forest soil. \emph{Journal of Geophysical Research}, \emph{114}(G4).
\url{https://doi.org/10.1029/2009jg000998}

\leavevmode\hypertarget{ref-Myers_Smith_2008}{}%
Myers-Smith, I. H., Harden, J. W., Wilmking, M., Fuller, C. C., McGuire,
A. D., \& Chapin, F. S. (2008). Wetland succession in a permafrost
collapse: Interactions between fire and thermokarst.
\emph{Biogeosciences}, \emph{5}(5), 1273--1286.
\url{https://doi.org/10.5194/bg-5-1273-2008}

\leavevmode\hypertarget{ref-Nagy_2018}{}%
Nagy, R. C., Porder, S., Brando, P., Davidson, E. A., Silva Figueira, A.
M. e, Neill, C., Riskin, S., \& Trumbore, S. (2018). Soil carbon
dynamics in soybean cropland and forests in mato grosso, brazil.
\emph{Journal of Geophysical Research: Biogeosciences}, \emph{123}(1),
18--31. \url{https://doi.org/10.1002/2017jg004269}

\leavevmode\hypertarget{ref-Nakagawa_2002}{}%
Nakagawa, F., Yoshida, N., Nojiri, Y., \& Makarov, V. (2002). Production
of methane from alasses in eastern siberia: Implications from its14C and
stable isotopic compositions. \emph{Global Biogeochemical Cycles},
\emph{16}(3), 14--11--14--15. \url{https://doi.org/10.1029/2000gb001384}

\leavevmode\hypertarget{ref-Natali_2015}{}%
Natali, S. M., Schuur, E. A. G., Mauritz, M., Schade, J. D., Celis, G.,
Crummer, K. G., Johnston, C., Krapek, J., Pegoraro, E., Salmon, V. G.,
\& Webb, E. E. (2015). Permafrost thaw and soil moisture driving CO 2
and CH 4 release from upland tundra. \emph{Journal of Geophysical
Research: Biogeosciences}, \emph{120}(3), 525--537.
\url{https://doi.org/10.1002/2014jg002872}

\leavevmode\hypertarget{ref-NATALI_2011}{}%
NATALI, S. M., SCHUUR, E. A. G., TRUCCO, C., PRIES, C. E. H., CRUMMER,
K. G., \& LOPEZ, A. F. B. (2011). Effects of experimental warming of
air, soil and permafrost on carbon balance in alaskan tundra.
\emph{Global Change Biology}, \emph{17}(3), 1394--1407.
\url{https://doi.org/10.1111/j.1365-2486.2010.02303.x}

\leavevmode\hypertarget{ref-Nave_2017}{}%
Nave, L. E., Drevnick, P. E., Heckman, K. A., Hofmeister, K. L.,
Veverica, T. J., \& Swanston, C. W. (2017). Soil hydrology, physical and
chemical properties and the distribution of carbon and mercury in a
postglacial lake-plain wetland. \emph{Geoderma}, \emph{305}, 40--52.
\url{https://doi.org/10.1016/j.geoderma.2017.05.035}

\leavevmode\hypertarget{ref-Neff_2006}{}%
Neff, J. C., Finlay, J. C., Zimov, S. A., Davydov, S. P., Carrasco, J.
J., Schuur, E. A. G., \& Davydova, A. I. (2006). Seasonal changes in the
age and structure of dissolved organic carbon in siberian rivers and
streams. \emph{Geophysical Research Letters}, \emph{33}(23).
\url{https://doi.org/10.1029/2006gl028222}

\leavevmode\hypertarget{ref-Negandhi_2013}{}%
Negandhi, K., Laurion, I., Whiticar, M. J., Galand, P. E., Xu, X., \&
Lovejoy, C. (2013). Small thaw ponds: An unaccounted source of methane
in the canadian high arctic. \emph{PLoS ONE}, \emph{8}(11), e78204.
\url{https://doi.org/10.1371/journal.pone.0078204}

\leavevmode\hypertarget{ref-Nichols_1967}{}%
Nichols, H. (1967). Pollen diagrams from sub-arctic central canada.
\emph{Science}, \emph{155}(3770), 1665--1668.
\url{https://doi.org/10.1126/science.155.3770.1665}

\leavevmode\hypertarget{ref-Nowinski_2010}{}%
Nowinski, N. S., Taneva, L., Trumbore, S. E., \& Welker, J. M. (2010).
Decomposition of old organic matter as a result of deeper active layers
in a snow depth manipulation experiment. \emph{Oecologia},
\emph{163}(3), 785--792. \url{https://doi.org/10.1007/s00442-009-1556-x}

\leavevmode\hypertarget{ref-Nowinski_2009}{}%
Nowinski, N. S., Trumbore, S. E., Jimenez, G., \& Fenn, M. E. (2009).
Alteration of belowground carbon dynamics by nitrogen addition in
southern california mixed conifer forests. \emph{Journal of Geophysical
Research: Biogeosciences}, \emph{114}(G2), n/a--n/a.
\url{https://doi.org/10.1029/2008jg000801}

\leavevmode\hypertarget{ref-O_Brien_1986}{}%
O\textquotesingleBrien, B. J. (1986). The use of natural and
anthropogenic 14C to investigate the dynamics of soil organic carbon.
\emph{Radiocarbon}, \emph{28}(2A), 358--362.
\url{https://doi.org/10.1017/s0033822200007463}

\leavevmode\hypertarget{ref-O_Brien_2011}{}%
O'Brien, S. L., Jastrow, J. D., McFarlane, K. J., Guilderson, T. P., \&
Gonzalez-Meler, M. A. (2011). Decadal cycling within long-lived carbon
pools revealed by dual isotopic analysis of mineral-associated soil
organic matter. \emph{Biogeochemistry}, \emph{112}(1-3), 111--125.
\url{https://doi.org/10.1007/s10533-011-9673-0}

\leavevmode\hypertarget{ref-O_Donnell_2014}{}%
O\textquotesingleDonnell, J. A., Aiken, G. R., Walvoord, M. A., Raymond,
P. A., Butler, K. D., Dornblaser, M. M., \& Heckman, K. (2014). Using
dissolved organic matter age and composition to detect permafrost thaw
in boreal watersheds of interior alaska. \emph{Journal of Geophysical
Research: Biogeosciences}, \emph{119}(11), 2155--2170.
\url{https://doi.org/10.1002/2014jg002695}

\leavevmode\hypertarget{ref-O_DONNELL_2010}{}%
O\textquotesingleDONNELL, J. A., HARDEN, J. W., McGUIRE, A. D.,
KANEVSKIY, M. Z., JORGENSON, M. T., \& XU, X. (2010). The effect of fire
and permafrost interactions on soil carbon accumulation in an upland
black spruce ecosystem of interior alaska: Implications for post-thaw
carbon loss. \emph{Global Change Biology}, \emph{17}(3), 1461--1474.
\url{https://doi.org/10.1111/j.1365-2486.2010.02358.x}

\leavevmode\hypertarget{ref-O_Donnell_2011}{}%
O'Donnell, J. A., Jorgenson, M. T., Harden, J. W., McGuire, A. D.,
Kanevskiy, M. Z., \& Wickland, K. P. (2011). The effects of permafrost
thaw on soil hydrologic, thermal, and carbon dynamics in an alaskan
peatland. \emph{Ecosystems}, \emph{15}(2), 213--229.
\url{https://doi.org/10.1007/s10021-011-9504-0}

\leavevmode\hypertarget{ref-Ohno_2017}{}%
Ohno, T., Heckman, K. A., Plante, A. F., Fernandez, I. J., \& Parr, T.
B. (2017). 14C mean residence time and its relationship with thermal
stability and molecular composition of soil organic matter: A case study
of deciduous and coniferous forest types. \emph{Geoderma}, \emph{308},
1--8. \url{https://doi.org/10.1016/j.geoderma.2017.08.023}

\leavevmode\hypertarget{ref-OKSANEN_2008}{}%
OKSANEN, P. O. (2008). Holocene development of the vaisjeäggi palsa
mire, finnish lapland. \emph{Boreas}, \emph{35}(1), 81--95.
\url{https://doi.org/10.1111/j.1502-3885.2006.tb01114.x}

\leavevmode\hypertarget{ref-Oksanen_2001}{}%
Oksanen, P. O., Kuhry, P., \& Alekseeva, R. N. (2001). Holocene
development of the rogovaya river peat plateau, european russian arctic.
\emph{The Holocene}, \emph{11}(1), 25--40.
\url{https://doi.org/10.1191/095968301675477157}

\leavevmode\hypertarget{ref-Oksanen_2005}{}%
Oksanen, P. O., Kuhry, P., \& Alekseeva, R. N. (2005). Holocene
development and permafrost history of the usinsk mire, northeast
european russia. \emph{Géographie Physique et Quaternaire},
\emph{57}(2-3), 169--187. \url{https://doi.org/10.7202/011312ar}

\leavevmode\hypertarget{ref-OVENDEN_2008}{}%
OVENDEN, L. (2008). Vegetation history of a polygonal peatland,
northern, yukon. \emph{Boreas}, \emph{11}(3), 209--224.
\url{https://doi.org/10.1111/j.1502-3885.1982.tb00715.x}

\leavevmode\hypertarget{ref-Page_2004}{}%
Page, S. E., Wűst, R. A. J., Weiss, D., Rieley, J. O., Shotyk, W., \&
Limin, S. H. (2004). A record of late pleistocene and holocene carbon
accumulation and climate change from an equatorial peat bog(Kalimantan,
indonesia): Implications for past, present and future carbon dynamics.
\emph{Journal of Quaternary Science}, \emph{19}(7), 625--635.
\url{https://doi.org/10.1002/jqs.884}

\leavevmode\hypertarget{ref-Panova_2010}{}%
Panova, N. K., Trofimova, S. S., Antipina, T. G., Zinoviev, E. V.,
Gilev, A. V., \& Erokhin, N. G. (2010). Holocene dynamics of vegetation
and ecological conditions in the southern yamal peninsula according to
the results of comprehensive analysis of a relict peat bog deposit.
\emph{Russian Journal of Ecology}, \emph{41}(1), 20--27.
\url{https://doi.org/10.1134/s1067413610010042}

\leavevmode\hypertarget{ref-Paul_1997}{}%
Paul, E. A., Follett, R. F., Leavitt, S. W., Halvorson, A., Peterson, G.
A., \& Lyon, D. J. (1997). Radiocarbon dating for determination of soil
organic matter pool sizes and dynamics. \emph{Soil Science Society of
America Journal}, \emph{61}(4), 1058--1067.
\url{https://doi.org/10.2136/sssaj1997.03615995006100040011x}

\leavevmode\hypertarget{ref-Paul_2001}{}%
Paul, E., Collins, H., \& Leavitt, S. (2001). Dynamics of resistant soil
carbon of midwestern agricultural soils measured by naturally occurring
14C abundance. \emph{Geoderma}, \emph{104}(3-4), 239--256.
\url{https://doi.org/10.1016/s0016-7061(01)00083-0}

\leavevmode\hypertarget{ref-https:ux2fux2fdoi.orgux2f10.5281ux2fzenodo.3370053}{}%
Pedron, S., Holden, S. R., Welker, J. M., Ziolkowski, L. A., Mortero,
G., Li, H., Walker, J., Xu, X., \& Czimczik, C. I. (2019). \emph{Soil
radiocarbon from moist acidic tussock and erect shrub tundra at toolik
field station}. Zenodo. \url{https://doi.org/10.5281/ZENODO.3370053}

\leavevmode\hypertarget{ref-Pegoraro_2020}{}%
Pegoraro, E. F., Mauritz, M. E., Ogle, K., Ebert, C. H., \& Schuur, E.
A. G. (2020). Lower soil moisture and deep soil temperatures in
thermokarst features increase old soil carbon loss after 10~years of
experimental permafrost warming. \emph{Global Change Biology},
\emph{27}(6), 1293--1308. \url{https://doi.org/10.1111/gcb.15481}

\leavevmode\hypertarget{ref-Pegoraro_2019}{}%
Pegoraro, E., Mauritz, M., Bracho, R., Ebert, C., Dijkstra, P., Hungate,
B. A., Konstantinidis, K. T., Luo, Y., Schädel, C., Tiedje, J. M., Zhou,
J., \& Schuur, E. A. (2019). Glucose addition increases the magnitude
and decreases the age of soil respired carbon in a long-term permafrost
incubation study. \emph{Soil Biology and Biochemistry}, \emph{129},
201--211. \url{https://doi.org/10.1016/j.soilbio.2018.10.009}

\leavevmode\hypertarget{ref-Pessenda_2001}{}%
Pessenda, L. C. R., Gouveia, S. E. M., \& Aravena, R. (2001).
Radiocarbon dating of total soil organic matter and humin fraction and
its comparison with 14C ages of fossil charcoal. \emph{Radiocarbon},
\emph{43}(2B), 595--601. \url{https://doi.org/10.1017/s0033822200041242}

\leavevmode\hypertarget{ref-Pessenda_1997}{}%
Pessenda, L. C. R., Gouveia, S. E. M., Aravena, R., Gomes, B. M.,
Boulet, R., \& Ribeiro, A. S. (1997). 14C dating and stable carbon
isotopes of soil organic matter in forestSavanna boundary areas in the
southern brazilian amazon region. \emph{Radiocarbon}, \emph{40}(2),
1013--1022. \url{https://doi.org/10.1017/s0033822200018981}

\leavevmode\hypertarget{ref-Pessenda_1996}{}%
Pessenda, L. C. R., Valencia, E. P. E., Camargo, P. B., Telles, E. C.
C., Martinelli, L. A., Cerri, C. C., Aravena, R., \& Rozanski, K.
(1996). Natural radiocarbon measurements in brazilian soils developed on
basic rocks. \emph{Radiocarbon}, \emph{38}(2), 203--208.
\url{https://doi.org/10.1017/s0033822200017574}

\leavevmode\hypertarget{ref-PETEET_2008}{}%
PETEET, D., ANDREEV, A., BARDEEN, W., \& MISTRETTA, F. (2008). Long-term
arctic peatland dynamics, vegetation and climate history of the pur-taz
region, western siberia. \emph{Boreas}, \emph{27}(2), 115--126.
\url{https://doi.org/10.1111/j.1502-3885.1998.tb00872.x}

\leavevmode\hypertarget{ref-P_rez_2006}{}%
Pérez, T., Garcia-Montiel, D., Trumbore, S., Tyler, S., Camargo, P. de,
Moreira, M., Piccolo, M., \& Cerri, C. (2006). NITROUS OXIDE
NITRIFICATION AND DENITRIFICATION15N ENRICHMENT FACTORS FROM AMAZON
FOREST SOILS. \emph{Ecological Applications}, \emph{16}(6), 2153--2167.
\href{https://doi.org/10.1890/1051-0761(2006)016\%5B2153:nonadn\%5D2.0.co;2}{https://doi.org/10.1890/1051-0761(2006)016{[}2153:nonadn{]}2.0.co;2}

\leavevmode\hypertarget{ref-Phillips_2013}{}%
Phillips, C. L., McFarlane, K. J., Risk, D., \& Desai, A. R. (2013).
Biological and physical influences on soil
\&lt\(\mathsemicolon\)sup\&gt\(\mathsemicolon\)14\&lt\(\mathsemicolon\)/sup\&gt\(\mathsemicolon\)CO\&lt\(\mathsemicolon\)sub\&gt\(\mathsemicolon\)2\&lt\(\mathsemicolon\)/sub\&gt\(\mathsemicolon\)
seasonal dynamics in a temperate hardwood forest. \emph{Biogeosciences},
\emph{10}(12), 7999--8012. \url{https://doi.org/10.5194/bg-10-7999-2013}

\leavevmode\hypertarget{ref-Phillips_1996}{}%
Phillips, S., \& Bustin, R. M. (1996). Sedimentology of the changuinola
peat deposit: Organic and clastic sedimentary response to punctuated
coastal subsidence. \emph{Geological Society of America Bulletin},
\emph{108}(7), 794--814.
\href{https://doi.org/10.1130/0016-7606(1996)108\%3C0794:sotcpd\%3E2.3.co;2}{https://doi.org/10.1130/0016-7606(1996)108\textless{}0794:sotcpd\textgreater{}2.3.co;2}

\leavevmode\hypertarget{ref-Posada_2011}{}%
Posada, J. M., \& Schuur, E. A. G. (2011). Relationships among
precipitation regime, nutrient availability, and carbon turnover in
tropical rain forests. \emph{Oecologia}, \emph{165}(3), 783--795.
\url{https://doi.org/10.1007/s00442-010-1881-0}

\leavevmode\hypertarget{ref-Hicks_Pries_2017}{}%
Pries, C. E. H., Castanha, C., Porras, R. C., \& Torn, M. S. (2017). The
whole-soil carbon flux in response to warming. \emph{Science},
\emph{355}(6332), 1420--1423.
\url{https://doi.org/10.1126/science.aal1319}

\leavevmode\hypertarget{ref-Hicks_Pries_2015}{}%
Pries, C. E. H., Logtestijn, R. S. P., Schuur, E. A. G., Natali, S. M.,
Cornelissen, J. H. C., Aerts, R., \& Dorrepaal, E. (2015). Decadal
warming causes a consistent and persistent shift from heterotrophic to
autotrophic respiration in contrasting permafrost ecosystems.
\emph{Global Change Biology}, \emph{21}(12), 4508--4519.
\url{https://doi.org/10.1111/gcb.13032}

\leavevmode\hypertarget{ref-Hicks_Pries_2011}{}%
Pries, C. E. H., Schuur, E. A. G., \& Crummer, K. G. (2011). Holocene
carbon stocks and carbon accumulation rates altered in soils undergoing
permafrost thaw. \emph{Ecosystems}, \emph{15}(1), 162--173.
\url{https://doi.org/10.1007/s10021-011-9500-4}

\leavevmode\hypertarget{ref-Hicks_Pries_2012}{}%
Pries, C. E. H., Schuur, E. A. G., \& Crummer, K. G. (2012). Thawing
permafrost increases old soil and autotrophic respiration in tundra:
Partitioning ecosystem respiration using \(\updelta\)13C and ∆14C.
\emph{Global Change Biology}, \emph{19}(2), 649--661.
\url{https://doi.org/10.1111/gcb.12058}

\leavevmode\hypertarget{ref-Quideau_2001}{}%
Quideau, S., Chadwick, O., Trumbore, S., Johnson-Maynard, J., Graham,
R., \& Anderson, M. (2001). Vegetation control on soil organic matter
dynamics. \emph{Organic Geochemistry}, \emph{32}(2), 247--252.
\url{https://doi.org/10.1016/s0146-6380(00)00171-6}

\leavevmode\hypertarget{ref-Rabbi_2013}{}%
Rabbi, S. M. F., Hua, Q., Daniel, H., Lockwood, P. V., Wilson, B. R., \&
Young, I. M. (2013). Mean residence time of soil organic carbon in
aggregates under contrasting land uses based on radiocarbon
measurements. \emph{Radiocarbon}, \emph{55}(1), 127--139.
\url{https://doi.org/10.2458/azu_js_rc.v55i1.16179}

\leavevmode\hypertarget{ref-Rasmussen_2018}{}%
Rasmussen, C., Throckmorton, H., Liles, G., Heckman, K., Meding, S., \&
Horwath, W. (2018). Controls on soil organic carbon partitioning and
stabilization in the california sierra nevada. \emph{Soil Systems},
\emph{2}(3), 41. \url{https://doi.org/10.3390/soilsystems2030041}

\leavevmode\hypertarget{ref-Rasmussen_2005}{}%
Rasmussen, C., Torn, M. S., \& Southard, R. J. (2005). Mineral
assemblage and aggregates control carbon dynamics in a california
conifer forest. \emph{Soil Science Society of America Journal},
\emph{69}(6), 1711--1721. \url{https://doi.org/10.2136/sssaj2005.0040}

\leavevmode\hypertarget{ref-Rasmussen_2010}{}%
Rasmussen, C., \& White, D. A. (2010). Vegetation effects on soil
organic carbon quality in an arid hyperthermic ecosystem. \emph{Soil
Science}, \emph{175}(9), 438--446.
\url{https://doi.org/10.1097/ss.0b013e3181f38400}

\leavevmode\hypertarget{ref-Reichenbach_2021}{}%
Reichenbach, M., Fiener, P., Garland, G., Griepentrog, M., Six, J., \&
Doetterl, S. (2021). \emph{The role of geochemistry in organic carbon
stabilization in tropical rainforest soils}.
\url{https://doi.org/10.5194/soil-2020-92}

\leavevmode\hypertarget{ref-Resh_2002}{}%
Resh, S. C., Binkley, D., \& Parrotta, J. A. (2002). Greater soil carbon
sequestration under nitrogen-fixing trees compared with eucalyptus
species. \emph{Ecosystems}, \emph{5}(3), 217--231.
\url{https://doi.org/10.1007/s10021-001-0067-3}

\leavevmode\hypertarget{ref-Rethemeyer_2005}{}%
Rethemeyer, J., Kramer, C., Gleixner, G., John, B., Yamashita, T.,
Flessa, H., Andersen, N., Nadeau, M.-J., \& Grootes, P. M. (2005).
Transformation of organic matter in agricultural soils: Radiocarbon
concentration versus soil depth. \emph{Geoderma}, \emph{128}(1-2),
94--105. \url{https://doi.org/10.1016/j.geoderma.2004.12.017}

\leavevmode\hypertarget{ref-Richter_1999}{}%
Richter, D. D., Markewitz, D., Trumbore, S. E., \& Wells, C. G. (1999).
Rapid accumulation and turnover of soil carbon in a re-establishing
forest. \emph{Nature}, \emph{400}(6739), 56--58.
\url{https://doi.org/10.1038/21867}

\leavevmode\hypertarget{ref-Robinson_2006}{}%
Robinson, S. D. (2006). Carbon accumulation in peatlands, southwestern
northwest territories, canada. \emph{Canadian Journal of Soil Science},
\emph{86}(Special Issue), 305--319.
\url{https://doi.org/10.4141/s05-086}

\leavevmode\hypertarget{ref-Rogers_2014}{}%
Rogers, B. M., Veraverbeke, S., Azzari, G., Czimczik, C. I., Holden, S.
R., Mouteva, G. O., Sedano, F., Treseder, K. K., \& Randerson, J. T.
(2014). Quantifying fire-wide carbon emissions in interior alaska using
field measurements and landsat imagery. \emph{Journal of Geophysical
Research: Biogeosciences}, \emph{119}(8), 1608--1629.
\url{https://doi.org/10.1002/2014jg002657}

\leavevmode\hypertarget{ref-Rumpel_2008}{}%
Rumpel, C., Chaplot, V., Chabbi, A., Largeau, C., \& Valentin, C.
(2008). Stabilisation of HF soluble and HCl resistant organic matter in
sloping tropical soils under slash and burn agriculture.
\emph{Geoderma}, \emph{145}(3-4), 347--354.
\url{https://doi.org/10.1016/j.geoderma.2008.04.001}

\leavevmode\hypertarget{ref-Rumpel_2002}{}%
Rumpel, C., Kögel-Knabner, I., \& Bruhn, F. (2002). Vertical
distribution, age, and chemical composition of organic carbon in two
forest soils of different pedogenesis. \emph{Organic Geochemistry},
\emph{33}(10), 1131--1142.
\url{https://doi.org/10.1016/s0146-6380(02)00088-8}

\leavevmode\hypertarget{ref-Ruwaimana_2020}{}%
Ruwaimana, M., Anshari, G. Z., Silva, L. C. R., \& Gavin, D. G. (2020).
The oldest extant tropical peatland in the world: A major carbon
reservoir for at least 470.167em000 years. \emph{Environmental Research
Letters}, \emph{15}(11), 114027.
\url{https://doi.org/10.1088/1748-9326/abb853}

\leavevmode\hypertarget{ref-Saiz_2015}{}%
Saiz, G., Bird, M., Wurster, C., Quesada, C. A., Ascough, P., Domingues,
T., Schrodt, F., Schwarz, M., Feldpausch, T. R., Veenendaal, E.,
Djagbletey, G., Jacobsen, G., Hien, F., Compaore, H., Diallo, A., \&
Lloyd, J. (2015). The influence of
c\&lt\(\mathsemicolon\)sub\&gt\(\mathsemicolon\)3\&lt\(\mathsemicolon\)/sub\&gt\(\mathsemicolon\)
and
c\&lt\(\mathsemicolon\)sub\&gt\(\mathsemicolon\)4\&lt\(\mathsemicolon\)/sub\&gt\(\mathsemicolon\)
vegetation on soil organic matter dynamics in contrasting semi-natural
tropical ecosystems. \emph{Biogeosciences}, \emph{12}(16), 5041--5059.
\url{https://doi.org/10.5194/bg-12-5041-2015}

\leavevmode\hypertarget{ref-Ouzilleau_Samson_2010}{}%
Samson, D. O., Bhiry, N., \& Lavoie, M. (2010). Late-holocene
palaeoecology of a polygonal peatland on the south shore of hudson
strait, northern québec, canada. \emph{The Holocene}, \emph{20}(4),
525--536. \url{https://doi.org/10.1177/0959683609356582}

\leavevmode\hypertarget{ref-Sanaiotti_2002}{}%
Sanaiotti, T. M., Martinelli, L. A., Victoria, R. L., Trumbore, S. E.,
\& Camargo, P. B. (2002). Past vegetation changes in amazon savannas
determined using carbon isotopes of soil organic matter1.
\emph{Biotropica}, \emph{34}(1), 2--16.
\url{https://doi.org/10.1111/j.1744-7429.2002.tb00237.x}

\leavevmode\hypertarget{ref-Sanderman_2017}{}%
Sanderman, J., Creamer, C., Baisden, W. T., Farrell, M., \& Fallon, S.
(2017). Greater soil carbon stocks and faster turnover rates with
increasing agricultural productivity. \emph{SOIL}, \emph{3}(1), 1--16.
\url{https://doi.org/10.5194/soil-3-1-2017}

\leavevmode\hypertarget{ref-Sangok_2020}{}%
Sangok, F. E., Sugiura, Y., Maie, N., Melling, L., Nakamura, T., Ikeya,
K., \& Watanabe, A. (2020). Variations in the rate of accumulation and
chemical structure of soil organic matter in a coastal peatland in
sarawak, malaysia. \emph{CATENA}, \emph{184}, 104244.
\url{https://doi.org/10.1016/j.catena.2019.104244}

\leavevmode\hypertarget{ref-Sannel_2008}{}%
Sannel, A. B. K., \& Kuhry, P. (2008). Long-term stability of permafrost
in subarctic peat plateaus, west-central canada. \emph{The Holocene},
\emph{18}(4), 589--601. \url{https://doi.org/10.1177/0959683608089658}

\leavevmode\hypertarget{ref-Scharpenseel_1973}{}%
Scharpenseel, H. W., \& Pietig, F. (1973). University of bonn natural
radiocarbon measurements v. \emph{Radiocarbon}, \emph{15}(1), 13--41.
\url{https://doi.org/10.1017/s0033822200058586}

\leavevmode\hypertarget{ref-Schimel_2011}{}%
Schimel, J. P., Wetterstedt, J. M., Holden, P. A., \& Trumbore, S. E.
(2011). Drying/rewetting cycles mobilize old c from deep soils from a
california annual grassland. \emph{Soil Biology and Biochemistry},
\emph{43}(5), 1101--1103.
\url{https://doi.org/10.1016/j.soilbio.2011.01.008}

\leavevmode\hypertarget{ref-Sch_ning_2006}{}%
Schöning, I., \& Kögel-Knabner, I. (2006). Chemical composition of young
and old carbon pools throughout cambisol and luvisol profiles under
forests. \emph{Soil Biology and Biochemistry}, \emph{38}(8), 2411--2424.
\url{https://doi.org/10.1016/j.soilbio.2006.03.005}

\leavevmode\hypertarget{ref-Schrumpf_2013}{}%
Schrumpf, M., Kaiser, K., Guggenberger, G., Persson, T., Kögel-Knabner,
I., \& Schulze, E.-D. (2013). Storage and stability of organic carbon in
soils as related to depth, occlusion within aggregates, and attachment
to minerals. \emph{Biogeosciences}, \emph{10}(3), 1675--1691.
\url{https://doi.org/10.5194/bg-10-1675-2013}

\leavevmode\hypertarget{ref-Schulze_2010}{}%
Schulze, K., Borken, W., \& Matzner, E. (2010). Dynamics of dissolved
organic 14C in throughfall and soil solution of a norway spruce forest.
\emph{Biogeochemistry}, \emph{106}(3), 461--473.
\url{https://doi.org/10.1007/s10533-010-9526-2}

\leavevmode\hypertarget{ref-Schulze_2009}{}%
Schulze, K., Borken, W., Muhr, J., \& Matzner, E. (2009). Stock,
turnover time and accumulation of organic matter in bulk and density
fractions of a podzol soil. \emph{European Journal of Soil Science},
\emph{60}(4), 567--577.
\url{https://doi.org/10.1111/j.1365-2389.2009.01134.x}

\leavevmode\hypertarget{ref-Schuur_2001}{}%
Schuur, E. A. G., Chadwick, O. A., \& Matson, P. A. (2001). CARBON
CYCLING AND SOIL CARBON STORAGE IN MESIC TO WET HAWAIIAN MONTANE
FORESTS. \emph{Ecology}, \emph{82}(11), 3182--3196.
\href{https://doi.org/10.1890/0012-9658(2001)082\%5B3182:ccascs\%5D2.0.co;2}{https://doi.org/10.1890/0012-9658(2001)082{[}3182:ccascs{]}2.0.co;2}

\leavevmode\hypertarget{ref-Schuur_2009}{}%
Schuur, E. A. G., Vogel, J. G., Crummer, K. G., Lee, H., Sickman, J. O.,
\& Osterkamp, T. E. (2009). The effect of permafrost thaw on old carbon
release and net carbon exchange from tundra. \emph{Nature},
\emph{459}(7246), 556--559. \url{https://doi.org/10.1038/nature08031}

\leavevmode\hypertarget{ref-Schuur_2005}{}%
Schuur, E. A., \& Trumbore, S. E. (2005). Partitioning sources of soil
respiration in boreal black spruce forest using radiocarbon.
\emph{Global Change Biology}, \emph{12}(2), 165--176.
\url{https://doi.org/10.1111/j.1365-2486.2005.01066.x}

\leavevmode\hypertarget{ref-Schwartz_1996}{}%
Schwartz, D., Foresta, H. de, Mariotti, A., Balesdent, J., Massimba, J.
P., \& Girardin, C. (1996). Present dynamics of the savanna-forest
boundary in the congolese mayombe: A pedological, botanical and isotopic
(13C and 14C) study. \emph{Oecologia}, \emph{106}(4), 516--524.
\url{https://doi.org/10.1007/bf00329710}

\leavevmode\hypertarget{ref-Shaw_2004}{}%
Shaw, D., Franklin, J., Bible, K., Klopatek, J., Freeman, E., Greene,
S., \& Parker, G. (2004). Ecological setting of the wind river
old-growth forest. \emph{Ecosystems}, \emph{7}(5).
\url{https://doi.org/10.1007/s10021-004-0135-6}

\leavevmode\hypertarget{ref-Shen_2001}{}%
Shen, C., Yi, W., Sun, Y., Xing, C., Yang, Y., Yuan, C., Li, Z., Peng,
S., An, Z., \& Liu, T. (2001). Distribution of 14C and 13C in forest
soils of the dinghushan biosphere reserve. \emph{Radiocarbon},
\emph{43}(2B), 671--678. \url{https://doi.org/10.1017/s0033822200041321}

\leavevmode\hypertarget{ref-Sierra_2013}{}%
Sierra, C. A., Jiménez, E. M., Reu, B., Peñuela, M. C., Thuille, A., \&
Quesada, C. A. (2013). Low vertical transfer rates of carbon inferred
from radiocarbon analysis in an amazon podzol. \emph{Biogeosciences},
\emph{10}(6), 3455--3464. \url{https://doi.org/10.5194/bg-10-3455-2013}

\leavevmode\hypertarget{ref-Sierra_2012}{}%
Sierra, C. A., Trumbore, S. E., Davidson, E. A., Frey, S. D., Savage, K.
E., \& Hopkins, F. M. (2012). Predicting decadal trends and transient
responses of radiocarbon storage and fluxes in a temperate forest soil.
\emph{Biogeosciences}, \emph{9}(8), 3013--3028.
\url{https://doi.org/10.5194/bg-9-3013-2012}

\leavevmode\hypertarget{ref-1987}{}%
\emph{Soils developed in granitic alluvium near merced, california}.
(1987). US Geological Survey. \url{https://doi.org/10.3133/b1590a}

\leavevmode\hypertarget{ref-Sollins_2009}{}%
Sollins, P., Kramer, M. G., Swanston, C., Lajtha, K., Filley, T.,
Aufdenkampe, A. K., Wagai, R., \& Bowden, R. D. (2009). Sequential
density fractionation across soils of contrasting mineralogy: Evidence
for both microbial- and mineral-controlled soil organic matter
stabilization. \emph{Biogeochemistry}, \emph{96}(1-3), 209--231.
\url{https://doi.org/10.1007/s10533-009-9359-z}

\leavevmode\hypertarget{ref-Sollins_2006}{}%
Sollins, P., Swanston, C., Kleber, M., Filley, T., Kramer, M., Crow, S.,
Caldwell, B. A., Lajtha, K., \& Bowden, R. (2006). Organic c and n
stabilization in a forest soil: Evidence from sequential density
fractionation. \emph{Soil Biology and Biochemistry}, \emph{38}(11),
3313--3324. \url{https://doi.org/10.1016/j.soilbio.2006.04.014}

\leavevmode\hypertarget{ref-Spielvogel_2008}{}%
Spielvogel, S., Prietzel, J., \& Kgel-Knabner, I. (2008). Soil organic
matter stabilization in acidic forest soils is preferential and soil
type-specific. \emph{European Journal of Soil Science}, \emph{59}(4),
674--692. \url{https://doi.org/10.1111/j.1365-2389.2008.01030.x}

\leavevmode\hypertarget{ref-Staub_1994}{}%
Staub, J. R., \& Esterle, J. S. (1994). Peat-accumulating depositional
systems of sarawak, east malaysia. \emph{Sedimentary Geology},
\emph{89}(1-2), 91--106.
\url{https://doi.org/10.1016/0037-0738(94)90085-x}

\leavevmode\hypertarget{ref-Staub_1993}{}%
Staub, J. R., \& Esterle, J. S. (1993). Provenance and sediment
dispersal in the rajang river delta/coastal plain system, sarawak, east
malaysia. \emph{Sedimentary Geology}, \emph{85}(1-4), 191--201.
\url{https://doi.org/10.1016/0037-0738(93)90083-h}

\leavevmode\hypertarget{ref-STAUB_2003}{}%
STAUB, J. R., \& GASTALDO, R. A. (2003). LATE QUATERNARY SEDIMENTATION
AND PEAT DEVELOPMENT IN THE RAJANG RIVER DELTA, SARAWAK, EAST MALAYSIA.
In \emph{Tropical deltas of southeast asia} (pp. 71--87). SEPM (Society
for Sedimentary Geology). \url{https://doi.org/10.2110/pec.03.76.0071}

\leavevmode\hypertarget{ref-Stephan_1983}{}%
Stephan, S., Berrier, J., Petre, A. D., Jeanson, C., Kooistra, M.,
Scharpenseel, H., \& Schiffmann, H. (1983). Characterization of in situ
organic matter constituents in vertisols from argentina, using
submicroscopic and cytochemical methods first report. \emph{Geoderma},
\emph{30}(1-4), 21--34.
\url{https://doi.org/10.1016/0016-7061(83)90054-x}

\leavevmode\hypertarget{ref-Stoner_2021}{}%
Stoner, S. W., Hoyt, A. M., Trumbore, S., Sierra, C. A., Schrumpf, M.,
Doetterl, S., Baisden, W. T., \& Schipper, L. A. (2021). Soil organic
matter turnover rates increase to match increased inputs in grazed
grasslands. \emph{Biogeochemistry}, \emph{156}(1), 145--160.
\url{https://doi.org/10.1007/s10533-021-00838-z}

\leavevmode\hypertarget{ref-Stout_1980}{}%
Stout, J. D., \& Goh, K. M. (1980). The use of radiocarbon to measure
the effects of earthworms on soil development. \emph{Radiocarbon},
\emph{22}(3), 892--896. \url{https://doi.org/10.1017/s0033822200010298}

\leavevmode\hypertarget{ref-Striegl_2007}{}%
Striegl, R. G., Dornblaser, M. M., Aiken, G. R., Wickland, K. P., \&
Raymond, P. A. (2007). Carbon export and cycling by the yukon, tanana,
and porcupine rivers, alaska, 2001-2005. \emph{Water Resources
Research}, \emph{43}(2). \url{https://doi.org/10.1029/2006wr005201}

\leavevmode\hypertarget{ref-Strobel_2019}{}%
Strobel, P., Kasper, T., Frenzel, P., Schittek, K., Quick, L., Meadows,
M., Mäusbacher, R., \& Haberzettl, T. (2019). Late quaternary
palaeoenvironmental change in the year-round rainfall zone of south
africa derived from peat sediments from vankervelsvlei. \emph{Quaternary
Science Reviews}, \emph{218}, 200--214.
\url{https://doi.org/10.1016/j.quascirev.2019.06.014}

\leavevmode\hypertarget{ref-Stubbins_2012}{}%
Stubbins, A., Hood, E., Raymond, P. A., Aiken, G. R., Sleighter, R. L.,
Hernes, P. J., Butman, D., Hatcher, P. G., Striegl, R. G., Schuster, P.,
Abdulla, H. A. N., Vermilyea, A. W., Scott, D. T., \& Spencer, R. G. M.
(2012). Anthropogenic aerosols as a source of ancient dissolved organic
matter in glaciers. \emph{Nature Geoscience}, \emph{5}(3), 198--201.
\url{https://doi.org/10.1038/ngeo1403}

\leavevmode\hypertarget{ref-Swanston_2005}{}%
Swanston, C. W., Torn, M. S., Hanson, P. J., Southon, J. R., Garten, C.
T., Hanlon, E. M., \& Ganio, L. (2005). Initial characterization of
processes of soil carbon stabilization using forest stand-level
radiocarbon enrichment. \emph{Geoderma}, \emph{128}(1-2), 52--62.
\url{https://doi.org/10.1016/j.geoderma.2004.12.015}

\leavevmode\hypertarget{ref-Swindles_2018}{}%
Swindles, G. T., Kelly, T. J., Roucoux, K. H., \& Lawson, I. T. (2018).
Response of testate amoebae to a late holocene ecosystem shift in an
amazonian peatland. \emph{European Journal of Protistology}, \emph{64},
13--19. \url{https://doi.org/10.1016/j.ejop.2018.03.002}

\leavevmode\hypertarget{ref-Swindles_2017}{}%
Swindles, G. T., Morris, P. J., Whitney, B., Galloway, J. M., Gałka, M.,
Gallego-Sala, A., Macumber, A. L., Mullan, D., Smith, M. W., Amesbury,
M. J., Roland, T. P., Sanei, H., Patterson, R. T., Sanderson, N., Parry,
L., Charman, D. J., Lopez, O., Valderamma, E., Watson, E. J., \ldots{}
Lähteenoja, O. (2017). Ecosystem state shifts during long-term
development of an amazonian peatland. \emph{Global Change Biology},
\emph{24}(2), 738--757. \url{https://doi.org/10.1111/gcb.13950}

\leavevmode\hypertarget{ref-Szymanski_2019}{}%
Szymanski, L. M., Sanford, G. R., Heckman, K. A., Jackson, R. D., \&
Marı'n-Spiotta, E. (2019). Conversion to bioenergy crops alters the
amount and age of microbially-respired soil carbon. \emph{Soil Biology
and Biochemistry}, \emph{128}, 35--44.
\url{https://doi.org/10.1016/j.soilbio.2018.08.025}

\leavevmode\hypertarget{ref-Tan_2013}{}%
Tan, W., Zhou, L., \& Liu, K. (2013). Soil aggregate fraction-based 14C
analysis and its application in the study of soil organic carbon
turnover under forests of different ages. \emph{Chinese Science
Bulletin}, \emph{58}(16), 1936--1947.
\url{https://doi.org/10.1007/s11434-012-5660-7}

\leavevmode\hypertarget{ref-de_Tapia_2005}{}%
Tapia, E. M. de, Rubio, I. D., Castro, J. G., Solleiro, E., \& Sedov, S.
(2005). Radiocarbon dates from soil profiles in the teotihuacán valley,
mexico: Indicators of geomorphological processes. \emph{Radiocarbon},
\emph{47}(1), 159--175. \url{https://doi.org/10.1017/s0033822200052279}

\leavevmode\hypertarget{ref-Taylor_2015}{}%
Taylor, A. J., Lai, C.-T., Hopkins, F. M., Wharton, S., Bible, K., Xu,
X., Phillips, C., Bush, S., \& Ehleringer, J. R. (2015).
Radiocarbon-based partitioning of soil respiration in an old-growth
coniferous forest. \emph{Ecosystems}, \emph{18}(3), 459--470.
\url{https://doi.org/10.1007/s10021-014-9839-4}

\leavevmode\hypertarget{ref-Tefs_2012}{}%
Tefs, C., \& Gleixner, G. (2012). Importance of root derived carbon for
soil organic matter storage in a temperate old-growth beech forest
evidence from c, n and 14C content. \emph{Forest Ecology and
Management}, \emph{263}, 131--137.
\url{https://doi.org/10.1016/j.foreco.2011.09.010}

\leavevmode\hypertarget{ref-Tegen_1996}{}%
Tegen, I., \& Dörr, H. (1996). 14C measurements of soil organic matter,
soil co2 and dissolved organic carbon (19871992). \emph{Radiocarbon},
\emph{38}(2), 247--251. \url{https://doi.org/10.1017/s0033822200017628}

\leavevmode\hypertarget{ref-https:ux2fux2fdoi.orgux2f10.3334ux2fornldaacux2f1025}{}%
TELLES, E., DE CAMARGO, P., MARTINELLI, L., TRUMBORE, S., DA COSTA, E.,
SANTOS, J., HIGUCHI, N., OLIVEIRA, R., \& MARKEWITZ, D. (2011).
\emph{LBA-eco cd-08 carbon isotopes in belowground carbon pools,
amazonas and para, brazil}. ORNL Distributed Active Archive Center.
\url{https://doi.org/10.3334/ORNLDAAC/1025}

\leavevmode\hypertarget{ref-Tifafi_2018}{}%
Tifafi, M., Camino-Serrano, M., Hatté, C., Morras, H., Moretti, L.,
Barbaro, S., Cornu, S., \& Guenet, B. (2018). The use of radiocarbon
\&lt\(\mathsemicolon\)sup\&gt\(\mathsemicolon\)14\&lt\(\mathsemicolon\)/sup\&gt\(\mathsemicolon\)C
to constrain carbon dynamics in the soil module of the land surface
model ORCHIDEE (SVN r5165). \emph{Geoscientific Model Development},
\emph{11}(12), 4711--4726.
\url{https://doi.org/10.5194/gmd-11-4711-2018}

\leavevmode\hypertarget{ref-Tonneijck_2006}{}%
Tonneijck, F. H., Plicht, J. van der, Jansen, B., Verstraten, J. M., \&
Hooghiemstra, H. (2006). Radiocarbon dating of soil organic matter
fractions in andosols in northern ecuador. \emph{Radiocarbon},
\emph{48}(3), 337--353. \url{https://doi.org/10.1017/s0033822200038790}

\leavevmode\hypertarget{ref-Torn_2002}{}%
Torn, M. S., Lapenis, A. G., Timofeev, A., Fischer, M. L., Babikov, B.
V., \& Harden, J. W. (2002). Organic carbon and carbon isotopes in
modern and 100-year-old-soil archives of the russian steppe.
\emph{Global Change Biology}, \emph{8}(10), 941--953.
\url{https://doi.org/10.1046/j.1365-2486.2002.00477.x}

\leavevmode\hypertarget{ref-Torn_1997}{}%
Torn, M. S., Trumbore, S. E., Chadwick, O. A., Vitousek, P. M., \&
Hendricks, D. M. (1997). Mineral control of soil organic carbon storage
and turnover. \emph{Nature}, \emph{389}(6647), 170--173.
\url{https://doi.org/10.1038/38260}

\leavevmode\hypertarget{ref-Torn_2005}{}%
Torn, M. S., Vitousek, P. M., \& Trumbore, S. E. (2005). The influence
of nutrient availability on soil organic matter turnover estimated by
incubations and radiocarbon modeling. \emph{Ecosystems}, \emph{8}(4),
352--372. \url{https://doi.org/10.1007/s10021-004-0259-8}

\leavevmode\hypertarget{ref-https:ux2fux2fdoi.orgux2f10.1594ux2fpangaea.863689}{}%
Treat, C. C., Jones, M. C., Camill, P., Gallego-Sala, A. V., Garneau,
M., Harden, J. W., Hugelius, G., Klein, E. S., Kokfelt, U., Kuhry, P.,
Loisel, J., Mathijssen, P. J. H., O'Donnell, J. A., Oksanen, P. O.,
Ronkainen, T. M., Sannel, A. B. K., Talbot, J., Tarnocai, C., \&
Väliranta, M. (2016). \emph{(Table s1) site locations of cores and
descriptions}. PANGAEA - Data Publisher for Earth \& Environmental
Science. \url{https://doi.org/10.1594/PANGAEA.863689}

\leavevmode\hypertarget{ref-Tremblay_2014}{}%
Tremblay, S., Bhiry, N., \& Lavoie, M. (2014). Long-term dynamics of a
palsa in the sporadic permafrost zone of northwestern quebec (canada).
\emph{Canadian Journal of Earth Sciences}, \emph{51}(5), 500--509.
\url{https://doi.org/10.1139/cjes-2013-0123}

\leavevmode\hypertarget{ref-Trumbore_1993}{}%
Trumbore, S. E. (1993). Comparison of carbon dynamics in tropical and
temperate soils using radiocarbon measurements. \emph{Global
Biogeochemical Cycles}, \emph{7}(2), 275--290.
\url{https://doi.org/10.1029/93gb00468}

\leavevmode\hypertarget{ref-Trumbore_1999}{}%
Trumbore, S. E., Bubier, J. L., Harden, J. W., \& Crill, P. M. (1999).
Carbon cycling in boreal wetlands: A comparison of three approaches.
\emph{Journal of Geophysical Research: Atmospheres}, \emph{104}(D22),
27673--27682. \url{https://doi.org/10.1029/1999jd900433}

\leavevmode\hypertarget{ref-Trumbore_1996}{}%
Trumbore, S. E., Chadwick, O. A., \& Amundson, R. (1996). Rapid exchange
between soil carbon and atmospheric carbon dioxide driven by temperature
change. \emph{Science}, \emph{272}(5260), 393--396.
\url{https://doi.org/10.1126/science.272.5260.393}

\leavevmode\hypertarget{ref-Trumbore_1995}{}%
Trumbore, S. E., Davidson, E. A., Camargo, P. B. de, Nepstad, D. C., \&
Martinelli, L. A. (1995). Belowground cycling of carbon in forests and
pastures of eastern amazonia. \emph{Global Biogeochemical Cycles},
\emph{9}(4), 515--528. \url{https://doi.org/10.1029/95gb02148}

\leavevmode\hypertarget{ref-Trumbore_1997}{}%
Trumbore, S. E., \& Harden, J. W. (1997). Accumulation and turnover of
carbon in organic and mineral soils of the BOREAS northern study area.
\emph{Journal of Geophysical Research: Atmospheres}, \emph{102}(D24),
28817--28830. \url{https://doi.org/10.1029/97jd02231}

\leavevmode\hypertarget{ref-https:ux2fux2fdoi.orgux2f10.5281ux2fzenodo.5168031}{}%
Trumbore, S., Lawrence, C., \& Khomo, L. (2021). \emph{Radiocarbon in
bulk and respired co2 from the cowlitz river chronosequence, washington,
usa}. Zenodo. \url{https://doi.org/10.5281/ZENODO.5168031}

\leavevmode\hypertarget{ref-Upton_2018}{}%
Upton, A., Vane, C. H., Girkin, N., Turner, B. L., \& Sjögersten, S.
(2018). Does litter input determine carbon storage and peat organic
chemistry in tropical peatlands? \emph{Geoderma}, \emph{326}, 76--87.
\url{https://doi.org/10.1016/j.geoderma.2018.03.030}

\leavevmode\hypertarget{ref-Vardy_1997}{}%
Vardy, S. R., Warner, B. G., \& Aravena, R. (1997). Holocene climate
effects on the development of a peatland on the tuktoyaktuk peninsula,
northwest territories. \emph{Quaternary Research}, \emph{47}(1),
90--104. \url{https://doi.org/10.1006/qres.1996.1869}

\leavevmode\hypertarget{ref-Vardy_1998}{}%
Vardy, S. R., Warner, B. G., \& Aravena, R. (1998). \emph{Climatic
Change}, \emph{40}(2), 285--313.
\url{https://doi.org/10.1023/a:1005473021115}

\leavevmode\hypertarget{ref-VARDY_2008}{}%
VARDY, S. R., WARNER, B. G., \& ASADA, T. (2008). Holocene environmental
change in two polygonal peatlands, south-central nunavut, canada.
\emph{Boreas}, \emph{34}(3), 324--334.
\url{https://doi.org/10.1111/j.1502-3885.2005.tb01104.x}

\leavevmode\hypertarget{ref-Vardy_2000}{}%
Vardy, S. R., Warner, B. G., Turunen, J., \& Aravena, R. (2000). Carbon
accumulation in permafrost peatlands in the northwest territories and
nunavut, canada. \emph{The Holocene}, \emph{10}(2), 273--280.
\url{https://doi.org/10.1191/095968300671749538}

\leavevmode\hypertarget{ref-Vaughn_2019}{}%
Vaughn, L. J. S., \& Torn, M. S. (2019). 14C evidence that millennial
and fast-cycling soil carbon are equally sensitive to warming.
\emph{Nature Climate Change}, \emph{9}(6), 467--471.
\url{https://doi.org/10.1038/s41558-019-0468-y}

\leavevmode\hypertarget{ref-Vaughn_2018}{}%
Vaughn, L. J. S., \& Torn, M. S. (2018). Radiocarbon measurements of
ecosystem respiration and soil pore-space
CO\&lt\(\mathsemicolon\)sub\&gt\(\mathsemicolon\)2\&lt\(\mathsemicolon\)/sub\&gt\(\mathsemicolon\)
in utqiaġvik (barrow), alaska. \emph{Earth System Science Data},
\emph{10}(4), 1943--1957.
\url{https://doi.org/10.5194/essd-10-1943-2018}

\leavevmode\hypertarget{ref-V_liranta_2003}{}%
Väliranta, M., Kaakinen, A., \& Kuhry, P. (2003). Holocene climate and
landscape evolution east of the pechora delta, east-european russian
arctic. \emph{Quaternary Research}, \emph{59}(3), 335--344.
\url{https://doi.org/10.1016/s0033-5894(03)00041-3}

\leavevmode\hypertarget{ref-van_der_Voort_2016}{}%
Voort, T. S. van der, Hagedorn, F., McIntyre, C., Zell, C., Walthert,
L., Schleppi, P., Feng, X., \& Eglinton, T. I. (2016). Variability in
\&lt\(\mathsemicolon\)sup\&gt\(\mathsemicolon\)14\&lt\(\mathsemicolon\)/sup\&gt\(\mathsemicolon\)C
contents of soil organic matter at the plot and regional scale across
climatic and geologic gradients. \emph{Biogeosciences}, \emph{13}(11),
3427--3439. \url{https://doi.org/10.5194/bg-13-3427-2016}

\leavevmode\hypertarget{ref-Wagai_2015}{}%
Wagai, R., Kajiura, M., Asano, M., \& Hiradate, S. (2015). Nature of
soil organo-mineral assemblage examined by sequential density
fractionation with and without sonication: Is allophanic soil different?
\emph{Geoderma}, \emph{241-242}, 295--305.
\url{https://doi.org/10.1016/j.geoderma.2014.11.028}

\leavevmode\hypertarget{ref-Wahlen_1989}{}%
Wahlen, M., Tanaka, N., Henry, R., Deck, B., Zeglen, J., Vogel, J. S.,
Southon, J., Shemesh, A., Fairbanks, R., \& Broecker, W. (1989).
Carbon-14 in methane sources and in atmospheric methane: The
contribution from fossil carbon. \emph{Science}, \emph{245}(4915),
286--290. \url{https://doi.org/10.1126/science.245.4915.286}

\leavevmode\hypertarget{ref-Waldron_2019}{}%
Waldron, S., Vihermaa, L., Evers, S., Garnett, M. H., Newton, J., \&
Henderson, A. C. G. (2019). C mobilisation in disturbed tropical peat
swamps: Old DOC can fuel the fluvial efflux of old carbon dioxide, but
site recovery can occur. \emph{Scientific Reports}, \emph{9}(1).
\url{https://doi.org/10.1038/s41598-019-46534-9}

\leavevmode\hypertarget{ref-Waldrop_2012}{}%
Waldrop, M., Harden, J., Turetsky, M., Petersen, D., McGuire, A.,
Briones, M., Churchill, A., Doctor, D., \& Pruett, L. (2012). Bacterial
and enchytraeid abundance accelerate soil carbon turnover along a
lowland vegetation gradient in interior alaska. \emph{Soil Biology and
Biochemistry}, \emph{50}, 188--198.
\url{https://doi.org/10.1016/j.soilbio.2012.02.032}

\leavevmode\hypertarget{ref-Walter_2008}{}%
Walter, K. M., Chanton, J. P., Chapin, F. S., Schuur, E. A. G., \&
Zimov, S. A. (2008). Methane production and bubble emissions from arctic
lakes: Isotopic implications for source pathways and ages. \emph{Journal
of Geophysical Research}, \emph{113}.
\url{https://doi.org/10.1029/2007jg000569}

\leavevmode\hypertarget{ref-WANG_2005}{}%
WANG, L., OUYANG, H., ZHOUM, C.-P., ZHANG, F., SONG, M.-H., \& TIAN,
Y.-Q. (2005). Soil organic matter dynamics along a vertical vegetation
gradient in the gongga mountain on the tibetan plateau. \emph{Journal of
Integrative Plant Biology}, \emph{47}(4), 411--420.
\url{https://doi.org/10.1111/j.1744-7909.2005.00085.x}

\leavevmode\hypertarget{ref-Wang_2018}{}%
Wang, X., Yoo, K., Mudd, S. M., Weinman, B., Gutknecht, J., \& Gabet, E.
J. (2018). Storage and export of soil carbon and mineral surface area
along an erosional gradient in the sierra nevada, california.
\emph{Geoderma}, \emph{321}, 151--163.
\url{https://doi.org/10.1016/j.geoderma.2018.02.008}

\leavevmode\hypertarget{ref-Wang_2000}{}%
Wang, Y., Amundson, R., \& Niu, X.-F. (2000). Seasonal and altitudinal
variation in decomposition of soil organic matter inferred from
radiocarbon measurements of soil CO2flux. \emph{Global Biogeochemical
Cycles}, \emph{14}(1), 199--211.
\url{https://doi.org/10.1029/1999gb900074}

\leavevmode\hypertarget{ref-Wang_1999}{}%
Wang, Y., Amundson, R., \& Trumbore, S. (1999). The impact of land use
change on c turnover in soils. \emph{Global Biogeochemical Cycles},
\emph{13}(1), 47--57. \url{https://doi.org/10.1029/1998gb900005}

\leavevmode\hypertarget{ref-Wang_1996}{}%
Wang, Y., Amundson, R., \& Trumbore, S. (1996). Radiocarbon dating of
soil organic matter. \emph{Quaternary Research}, \emph{45}(3), 282--288.
\url{https://doi.org/10.1006/qres.1996.0029}

\leavevmode\hypertarget{ref-WERNER_2010}{}%
WERNER, K., TARASOV, P. E., ANDREEV, A. A., MÜLLER, S., KIENAST, F.,
ZECH, M., ZECH, W., \& DIEKMANN, B. (2010). A 12.5-kyr history of
vegetation dynamics and mire development with evidence of younger dryas
larch presence in the verkhoyansk mountains, east siberia, russia.
\emph{Boreas}, \emph{39}(1), 56--68.
\url{https://doi.org/10.1111/j.1502-3885.2009.00116.x}

\leavevmode\hypertarget{ref-Wooller_2007}{}%
Wooller, M. J., Morgan, R., Fowell, S., Behling, H., \& Fogel, M.
(2007). A multiproxy peat record of holocene mangrove palaeoecology from
twin cays, belize. \emph{The Holocene}, \emph{17}(8), 1129--1139.
\url{https://doi.org/10.1177/0959683607082553}

\leavevmode\hypertarget{ref-Wunderlich_2012}{}%
Wunderlich, S., \& Borken, W. (2012). Partitioning of soil
CO\&lt\(\mathsemicolon\)sub\&gt\(\mathsemicolon\)2\&lt\(\mathsemicolon\)/sub\&gt\(\mathsemicolon\)
efflux in un-manipulated and experimentally flooded plots of a temperate
fen. \emph{Biogeosciences}, \emph{9}(8), 3477--3489.
\url{https://doi.org/10.5194/bg-9-3477-2012}

\leavevmode\hypertarget{ref-Yu_2003}{}%
Yu, Z., Campbell, I. D., Campbell, C., Vitt, D. H., Bond, G. C., \&
Apps, M. J. (2003). Carbon sequestration in western canadian peat highly
sensitive to holocene wet-dry climate cycles at millennial timescales.
\emph{The Holocene}, \emph{13}(6), 801--808.
\url{https://doi.org/10.1191/0959683603hl667ft}

\leavevmode\hypertarget{ref-Yulianto_2005}{}%
Yulianto, E., Rahardjo, A., Noeradi, D., Siregar, D., \& Hirakawa, K.
(2005). A holocene pollen record of vegetation and coastal environmental
changes in the coastal swamp forest at batulicin, south kalimantan,
indonesia. \emph{Journal of Asian Earth Sciences}, \emph{25}(1), 1--8.
\url{https://doi.org/10.1016/j.jseaes.2004.01.005}

\leavevmode\hypertarget{ref-Zazovskaya_2016}{}%
Zazovskaya, E., Mergelov, N., Shishkov, V., Dolgikh, A., Miamin, V.,
Cherkinsky, A., \& Goryachkin, S. (2016). Radiocarbon age of soils in
oases of east antarctica. \emph{Radiocarbon}, \emph{59}(2), 489--503.
\url{https://doi.org/10.1017/rdc.2016.75}

\leavevmode\hypertarget{ref-Zhang_2018}{}%
Zhang, H., Gallego-Sala, A. V., Amesbury, M. J., Charman, D. J., Piilo,
S. R., \& Väliranta, M. M. (2018). Inconsistent response of arctic
permafrost peatland carbon accumulation to warm climate phases.
\emph{Global Biogeochemical Cycles}, \emph{32}(10), 1605--1620.
\url{https://doi.org/10.1029/2018gb005980}

\leavevmode\hypertarget{ref-Zibulski_2013}{}%
Zibulski, R., Herzschuh, U., Pestryakova, L. A., Wolter, J., Müller, S.,
Schilling, N., Wetterich, S., Schirrmeister, L., \& Tian, F. (2013).
\emph{River flooding as a driver of polygon dynamics: Modern vegetation
data and a millennial peat record from the anabar river lowlands (arctic
siberia)}. \url{https://doi.org/10.5194/bgd-10-4067-2013}

\leavevmode\hypertarget{ref-Zimmermann_2001}{}%
Zimmermann, C., \& Lavoie, C. (2001). A paleoecological analysis of a
southern permafrost peatland, charlevoix, quebec. \emph{Canadian Journal
of Earth Sciences}, \emph{38}(6), 909--919.
\url{https://doi.org/10.1139/e00-110}

\leavevmode\hypertarget{ref-Zimov_1997}{}%
Zimov, S. A., Voropaev, Y. V., Semiletov, I. P., Davidov, S. P.,
Prosiannikov, S. F., Chapin, F. S., Chapin, M. C., Trumbore, S., \&
Tyler, S. (1997). North siberian lakes: A methane source fueled by
pleistocene carbon. \emph{Science}, \emph{277}(5327), 800--802.
\url{https://doi.org/10.1126/science.277.5327.800}

\leavevmode\hypertarget{ref-Zoltai_1985}{}%
Zoltai, S. C., \& Johnson, J. D. (1985). Development of a treed bog
island in a minerotrophic fen. \emph{Canadian Journal of Botany},
\emph{63}(6), 1076--1085. \url{https://doi.org/10.1139/b85-148}

\leavevmode\hypertarget{ref-Zoltai_1975}{}%
Zoltai, S. C., \& Tarnocai, C. (1975). Perennially frozen peatlands in
the western arctic and subarctic of canada. \emph{Canadian Journal of
Earth Sciences}, \emph{12}(1), 28--43.
\url{https://doi.org/10.1139/e75-004}

\leavevmode\hypertarget{ref-Zuidhoff_2000}{}%
Zuidhoff, F. S., \& Kolstrup, E. (2000). Changes in palsa distribution
in relation to climate change in laivadalen, northern sweden, especially
1960-1997. \emph{Permafrost and Periglacial Processes}, \emph{11}(1),
55--69.
\href{https://doi.org/10.1002/(sici)1099-1530(200001/03)11:1\%3C55::aid-ppp338\%3E3.0.co;2-t}{https://doi.org/10.1002/(sici)1099-1530(200001/03)11:1\textless{}55::aid-ppp338\textgreater{}3.0.co;2-t}

\end{document}
